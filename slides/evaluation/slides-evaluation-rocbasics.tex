\documentclass[11pt,compress,t,notes=noshow, xcolor=table]{beamer}

\usepackage[]{graphicx}\usepackage[]{color}
% maxwidth is the original width if it is less than linewidth
% otherwise use linewidth (to make sure the graphics do not exceed the margin)
\makeatletter
\def\maxwidth{ %
  \ifdim\Gin@nat@width>\linewidth
    \linewidth
  \else
    \Gin@nat@width
  \fi
}
\makeatother



\definecolor{fgcolor}{rgb}{0.345, 0.345, 0.345}
\newcommand{\hlnum}[1]{\textcolor[rgb]{0.686,0.059,0.569}{#1}}%
\newcommand{\hlstr}[1]{\textcolor[rgb]{0.192,0.494,0.8}{#1}}%
\newcommand{\hlcom}[1]{\textcolor[rgb]{0.678,0.584,0.686}{\textit{#1}}}%
\newcommand{\hlopt}[1]{\textcolor[rgb]{0,0,0}{#1}}%
\newcommand{\hlstd}[1]{\textcolor[rgb]{0.345,0.345,0.345}{#1}}%
\newcommand{\hlkwa}[1]{\textcolor[rgb]{0.161,0.373,0.58}{\textbf{#1}}}%
\newcommand{\hlkwb}[1]{\textcolor[rgb]{0.69,0.353,0.396}{#1}}%
\newcommand{\hlkwc}[1]{\textcolor[rgb]{0.333,0.667,0.333}{#1}}%
\newcommand{\hlkwd}[1]{\textcolor[rgb]{0.737,0.353,0.396}{\textbf{#1}}}%
\let\hlipl\hlkwb

\usepackage{framed}
\makeatletter
\newenvironment{kframe}{%
 \def\at@end@of@kframe{}%
 \ifinner\ifhmode%
  \def\at@end@of@kframe{\end{minipage}}%
  \begin{minipage}{\columnwidth}%
 \fi\fi%
 \def\FrameCommand##1{\hskip\@totalleftmargin \hskip-\fboxsep
 \colorbox{shadecolor}{##1}\hskip-\fboxsep
     % There is no \\@totalrightmargin, so:
     \hskip-\linewidth \hskip-\@totalleftmargin \hskip\columnwidth}%
 \MakeFramed {\advance\hsize-\width
   \@totalleftmargin\z@ \linewidth\hsize
   \@setminipage}}%
 {\par\unskip\endMakeFramed%
 \at@end@of@kframe}
\makeatother

\definecolor{shadecolor}{rgb}{.97, .97, .97}
\definecolor{messagecolor}{rgb}{0, 0, 0}
\definecolor{warningcolor}{rgb}{1, 0, 1}
\definecolor{errorcolor}{rgb}{1, 0, 0}
\newenvironment{knitrout}{}{} % an empty environment to be redefined in TeX

\usepackage{alltt}
\newcommand{\SweaveOpts}[1]{}  % do not interfere with LaTeX
\newcommand{\SweaveInput}[1]{} % because they are not real TeX commands
\newcommand{\Sexpr}[1]{}       % will only be parsed by R
\newcommand{\xmark}{\ding{55}}%


\usepackage[english]{babel}
\usepackage[utf8]{inputenc}

\usepackage{dsfont}
\usepackage{verbatim}
\usepackage{amsmath}
\usepackage{amsfonts}
\usepackage{amssymb}
\usepackage{bm}
\usepackage{csquotes}
\usepackage{multirow}
\usepackage{longtable}
\usepackage{booktabs}
\usepackage{enumerate}
\usepackage[absolute,overlay]{textpos}
\usepackage{psfrag}
\usepackage{algorithm}
\usepackage{algpseudocode}
\usepackage{eqnarray}
\usepackage{arydshln}
\usepackage{tabularx}
\usepackage{placeins}
\usepackage{tikz}
\usepackage{setspace}
\usepackage{colortbl}
\usepackage{mathtools}
\usepackage{wrapfig}
\usepackage{bm}
\usepackage{amsmath}
\usepackage{pifont}

\usetikzlibrary{shapes,arrows,automata,positioning,calc,chains,trees, shadows}
\tikzset{
  %Define standard arrow tip
  >=stealth',
  %Define style for boxes
  punkt/.style={
    rectangle,
    rounded corners,
    draw=black, very thick,
    text width=6.5em,
    minimum height=2em,
    text centered},
  % Define arrow style
  pil/.style={
    ->,
    thick,
    shorten <=2pt,
    shorten >=2pt,}
}

\usepackage{subfig}

% Defines macros and environments
\usepackage{../../style/lmu-lecture}


\let\code=\texttt
\let\proglang=\textsf

\setkeys{Gin}{width=0.9\textwidth}

\setbeamertemplate{frametitle}{\expandafter\uppercase\expandafter\insertframetitle}

% This file is included in slides and exercises

% Rarely used fontstyle for R packages, used only in 
% - forests/slides-forests-benchmark.tex
% - exercises/single-exercises/methods_l_1.Rnw
% - slides/cart/attic/slides_extra_trees.Rnw
\newcommand{\pkg}[1]{{\fontseries{b}\selectfont #1}}

% Spacing helpers, used often (mostly in exercises for \dlz)
\newcommand{\lz}{\vspace{0.5cm}} % vertical space (used often in slides)
\newcommand{\dlz}{\vspace{1cm}}  % double vertical space (used often in exercises, never in slides)
\newcommand{\oneliner}[1] % Oneliner for important statements, used e.g. in iml, algods
{\begin{block}{}\begin{center}\begin{Large}#1\end{Large}\end{center}\end{block}}

% Don't know if this is used or needed, remove?
% textcolor that works in mathmode
% https://tex.stackexchange.com/a/261480
% Used e.g. in forests/slides-forests-bagging.tex
% [...] \textcolor{blue}{\tfrac{1}{M}\sum^M_{m} [...]
% \makeatletter
% \renewcommand*{\@textcolor}[3]{%
%   \protect\leavevmode
%   \begingroup
%     \color#1{#2}#3%
%   \endgroup
% }
% \makeatother





% dependencies: amsmath, amssymb, dsfont
% math spaces
\ifdefined\N
\renewcommand{\N}{\mathds{N}} % N, naturals
\else \newcommand{\N}{\mathds{N}} \fi
\newcommand{\Z}{\mathds{Z}} % Z, integers
\newcommand{\Q}{\mathds{Q}} % Q, rationals
\newcommand{\R}{\mathds{R}} % R, reals
\ifdefined\C
\renewcommand{\C}{\mathds{C}} % C, complex
\else \newcommand{\C}{\mathds{C}} \fi
\newcommand{\continuous}{\mathcal{C}} % C, space of continuous functions
\newcommand{\M}{\mathcal{M}} % machine numbers
\newcommand{\epsm}{\epsilon_m} % maximum error

% counting / finite sets
\newcommand{\setzo}{\{0, 1\}} % set 0, 1
\newcommand{\setmp}{\{-1, +1\}} % set -1, 1
\newcommand{\unitint}{[0, 1]} % unit interval

% basic math stuff
\newcommand{\xt}{\tilde x} % x tilde
\DeclareMathOperator*{\argmax}{arg\,max} % argmax
\DeclareMathOperator*{\argmin}{arg\,min} % argmin
\newcommand{\argminlim}{\mathop{\mathrm{arg\,min}}\limits} % argmax with limits
\newcommand{\argmaxlim}{\mathop{\mathrm{arg\,max}}\limits} % argmin with limits
\newcommand{\sign}{\operatorname{sign}} % sign, signum
\newcommand{\I}{\mathbb{I}} % I, indicator
\newcommand{\order}{\mathcal{O}} % O, order
\newcommand{\bigO}{\mathcal{O}} % Big-O Landau
\newcommand{\littleo}{{o}} % Little-o Landau
\newcommand{\pd}[2]{\frac{\partial{#1}}{\partial #2}} % partial derivative
\newcommand{\floorlr}[1]{\left\lfloor #1 \right\rfloor} % floor
\newcommand{\ceillr}[1]{\left\lceil #1 \right\rceil} % ceiling
\newcommand{\indep}{\perp \!\!\! \perp} % independence symbol

% sums and products
\newcommand{\sumin}{\sum\limits_{i=1}^n} % summation from i=1 to n
\newcommand{\sumim}{\sum\limits_{i=1}^m} % summation from i=1 to m
\newcommand{\sumjn}{\sum\limits_{j=1}^n} % summation from j=1 to p
\newcommand{\sumjp}{\sum\limits_{j=1}^p} % summation from j=1 to p
\newcommand{\sumik}{\sum\limits_{i=1}^k} % summation from i=1 to k
\newcommand{\sumkg}{\sum\limits_{k=1}^g} % summation from k=1 to g
\newcommand{\sumjg}{\sum\limits_{j=1}^g} % summation from j=1 to g
\newcommand{\summM}{\sum\limits_{m=1}^M} % summation from m=1 to M
\newcommand{\meanin}{\frac{1}{n} \sum\limits_{i=1}^n} % mean from i=1 to n
\newcommand{\meanim}{\frac{1}{m} \sum\limits_{i=1}^m} % mean from i=1 to n
\newcommand{\meankg}{\frac{1}{g} \sum\limits_{k=1}^g} % mean from k=1 to g
\newcommand{\meanmM}{\frac{1}{M} \sum\limits_{m=1}^M} % mean from m=1 to M
\newcommand{\prodin}{\prod\limits_{i=1}^n} % product from i=1 to n
\newcommand{\prodkg}{\prod\limits_{k=1}^g} % product from k=1 to g
\newcommand{\prodjp}{\prod\limits_{j=1}^p} % product from j=1 to p

% linear algebra
\newcommand{\one}{\bm{1}} % 1, unitvector
\newcommand{\zero}{\mathbf{0}} % 0-vector
\newcommand{\id}{\bm{I}} % I, identity
\newcommand{\diag}{\operatorname{diag}} % diag, diagonal
\newcommand{\trace}{\operatorname{tr}} % tr, trace
\newcommand{\spn}{\operatorname{span}} % span
\newcommand{\scp}[2]{\left\langle #1, #2 \right\rangle} % <.,.>, scalarproduct
\newcommand{\mat}[1]{\begin{pmatrix} #1 \end{pmatrix}} % short pmatrix command
\newcommand{\Amat}{\mathbf{A}} % matrix A
\newcommand{\Deltab}{\mathbf{\Delta}} % error term for vectors

% basic probability + stats
\renewcommand{\P}{\mathds{P}} % P, probability
\newcommand{\E}{\mathds{E}} % E, expectation
\newcommand{\var}{\mathsf{Var}} % Var, variance
\newcommand{\cov}{\mathsf{Cov}} % Cov, covariance
\newcommand{\corr}{\mathsf{Corr}} % Corr, correlation
\newcommand{\normal}{\mathcal{N}} % N of the normal distribution
\newcommand{\iid}{\overset{i.i.d}{\sim}} % dist with i.i.d superscript
\newcommand{\distas}[1]{\overset{#1}{\sim}} % ... is distributed as ...

% machine learning
\newcommand{\Xspace}{\mathcal{X}} % X, input space
\newcommand{\Yspace}{\mathcal{Y}} % Y, output space
\newcommand{\Zspace}{\mathcal{Z}} % Z, space of sampled datapoints
\newcommand{\nset}{\{1, \ldots, n\}} % set from 1 to n
\newcommand{\pset}{\{1, \ldots, p\}} % set from 1 to p
\newcommand{\gset}{\{1, \ldots, g\}} % set from 1 to g
\newcommand{\Pxy}{\mathbb{P}_{xy}} % P_xy
\newcommand{\Exy}{\mathbb{E}_{xy}} % E_xy: Expectation over random variables xy
\newcommand{\xv}{\mathbf{x}} % vector x (bold)
\newcommand{\xtil}{\tilde{\mathbf{x}}} % vector x-tilde (bold)
\newcommand{\yv}{\mathbf{y}} % vector y (bold)
\newcommand{\xy}{(\xv, y)} % observation (x, y)
\newcommand{\xvec}{\left(x_1, \ldots, x_p\right)^\top} % (x1, ..., xp)
\newcommand{\Xmat}{\mathbf{X}} % Design matrix
\newcommand{\allDatasets}{\mathds{D}} % The set of all datasets
\newcommand{\allDatasetsn}{\mathds{D}_n}  % The set of all datasets of size n
\newcommand{\D}{\mathcal{D}} % D, data
\newcommand{\Dn}{\D_n} % D_n, data of size n
\newcommand{\Dtrain}{\mathcal{D}_{\text{train}}} % D_train, training set
\newcommand{\Dtest}{\mathcal{D}_{\text{test}}} % D_test, test set
\newcommand{\xyi}[1][i]{\left(\xv^{(#1)}, y^{(#1)}\right)} % (x^i, y^i), i-th observation
\newcommand{\Dset}{\left( \xyi[1], \ldots, \xyi[n]\right)} % {(x1,y1)), ..., (xn,yn)}, data
\newcommand{\defAllDatasetsn}{(\Xspace \times \Yspace)^n} % Def. of the set of all datasets of size n
\newcommand{\defAllDatasets}{\bigcup_{n \in \N}(\Xspace \times \Yspace)^n} % Def. of the set of all datasets
\newcommand{\xdat}{\left\{ \xv^{(1)}, \ldots, \xv^{(n)}\right\}} % {x1, ..., xn}, input data
\newcommand{\ydat}{\left\{ \yv^{(1)}, \ldots, \yv^{(n)}\right\}} % {y1, ..., yn}, input data
\newcommand{\yvec}{\left(y^{(1)}, \hdots, y^{(n)}\right)^\top} % (y1, ..., yn), vector of outcomes
\newcommand{\greekxi}{\xi} % Greek letter xi
\renewcommand{\xi}[1][i]{\xv^{(#1)}} % x^i, i-th observed value of x
\newcommand{\yi}[1][i]{y^{(#1)}} % y^i, i-th observed value of y
\newcommand{\xivec}{\left(x^{(i)}_1, \ldots, x^{(i)}_p\right)^\top} % (x1^i, ..., xp^i), i-th observation vector
\newcommand{\xj}{\xv_j} % x_j, j-th feature
\newcommand{\xjvec}{\left(x^{(1)}_j, \ldots, x^{(n)}_j\right)^\top} % (x^1_j, ..., x^n_j), j-th feature vector
\newcommand{\phiv}{\mathbf{\phi}} % Basis transformation function phi
\newcommand{\phixi}{\mathbf{\phi}^{(i)}} % Basis transformation of xi: phi^i := phi(xi)

%%%%%% ml - models general
\newcommand{\lamv}{\bm{\lambda}} % lambda vector, hyperconfiguration vector
\newcommand{\Lam}{\bm{\Lambda}}	 % Lambda, space of all hpos
% Inducer / Inducing algorithm
\newcommand{\preimageInducer}{\left(\defAllDatasets\right)\times\Lam} % Set of all datasets times the hyperparameter space
\newcommand{\preimageInducerShort}{\allDatasets\times\Lam} % Set of all datasets times the hyperparameter space
% Inducer / Inducing algorithm
\newcommand{\ind}{\mathcal{I}} % Inducer, inducing algorithm, learning algorithm

% continuous prediction function f
\newcommand{\ftrue}{f_{\text{true}}}  % True underlying function (if a statistical model is assumed)
\newcommand{\ftruex}{\ftrue(\xv)} % True underlying function (if a statistical model is assumed)
\newcommand{\fx}{f(\xv)} % f(x), continuous prediction function
\newcommand{\fdomains}{f: \Xspace \rightarrow \R^g} % f with domain and co-domain
\newcommand{\Hspace}{\mathcal{H}} % hypothesis space where f is from
\newcommand{\fbayes}{f^{\ast}} % Bayes-optimal model
\newcommand{\fxbayes}{f^{\ast}(\xv)} % Bayes-optimal model
\newcommand{\fkx}[1][k]{f_{#1}(\xv)} % f_j(x), discriminant component function
\newcommand{\fh}{\hat{f}} % f hat, estimated prediction function
\newcommand{\fxh}{\fh(\xv)} % fhat(x)
\newcommand{\fxt}{f(\xv ~|~ \thetav)} % f(x | theta)
\newcommand{\fxi}{f\left(\xv^{(i)}\right)} % f(x^(i))
\newcommand{\fxih}{\hat{f}\left(\xv^{(i)}\right)} % f(x^(i))
\newcommand{\fxit}{f\left(\xv^{(i)} ~|~ \thetav\right)} % f(x^(i) | theta)
\newcommand{\fhD}{\fh_{\D}} % fhat_D, estimate of f based on D
\newcommand{\fhDtrain}{\fh_{\Dtrain}} % fhat_Dtrain, estimate of f based on D
\newcommand{\fhDnlam}{\fh_{\Dn, \lamv}} %model learned on Dn with hp lambda
\newcommand{\fhDlam}{\fh_{\D, \lamv}} %model learned on D with hp lambda
\newcommand{\fhDnlams}{\fh_{\Dn, \lamv^\ast}} %model learned on Dn with optimal hp lambda
\newcommand{\fhDlams}{\fh_{\D, \lamv^\ast}} %model learned on D with optimal hp lambda

% discrete prediction function h
\newcommand{\hx}{h(\xv)} % h(x), discrete prediction function
\newcommand{\hh}{\hat{h}} % h hat
\newcommand{\hxh}{\hat{h}(\xv)} % hhat(x)
\newcommand{\hxt}{h(\xv | \thetav)} % h(x | theta)
\newcommand{\hxi}{h\left(\xi\right)} % h(x^(i))
\newcommand{\hxit}{h\left(\xi ~|~ \thetav\right)} % h(x^(i) | theta)
\newcommand{\hbayes}{h^{\ast}} % Bayes-optimal classification model
\newcommand{\hxbayes}{h^{\ast}(\xv)} % Bayes-optimal classification model

% yhat
\newcommand{\yh}{\hat{y}} % yhat for prediction of target
\newcommand{\yih}{\hat{y}^{(i)}} % yhat^(i) for prediction of ith targiet
\newcommand{\resi}{\yi- \yih}

% theta
\newcommand{\thetah}{\hat{\theta}} % theta hat
\newcommand{\thetav}{\bm{\theta}} % theta vector
\newcommand{\thetavh}{\bm{\hat\theta}} % theta vector hat
\newcommand{\thetat}[1][t]{\thetav^{[#1]}} % theta^[t] in optimization
\newcommand{\thetatn}[1][t]{\thetav^{[#1 +1]}} % theta^[t+1] in optimization
\newcommand{\thetahDnlam}{\thetavh_{\Dn, \lamv}} %theta learned on Dn with hp lambda
\newcommand{\thetahDlam}{\thetavh_{\D, \lamv}} %theta learned on D with hp lambda
\newcommand{\mint}{\min_{\thetav \in \Theta}} % min problem theta
\newcommand{\argmint}{\argmin_{\thetav \in \Theta}} % argmin theta

% densities + probabilities
% pdf of x
\newcommand{\pdf}{p} % p
\newcommand{\pdfx}{p(\xv)} % p(x)
\newcommand{\pixt}{\pi(\xv~|~ \thetav)} % pi(x|theta), pdf of x given theta
\newcommand{\pixit}[1][i]{\pi\left(\xi[#1] ~|~ \thetav\right)} % pi(x^i|theta), pdf of x given theta
\newcommand{\pixii}[1][i]{\pi\left(\xi[#1]\right)} % pi(x^i), pdf of i-th x

% pdf of (x, y)
\newcommand{\pdfxy}{p(\xv,y)} % p(x, y)
\newcommand{\pdfxyt}{p(\xv, y ~|~ \thetav)} % p(x, y | theta)
\newcommand{\pdfxyit}{p\left(\xi, \yi ~|~ \thetav\right)} % p(x^(i), y^(i) | theta)

% pdf of x given y
\newcommand{\pdfxyk}[1][k]{p(\xv | y= #1)} % p(x | y = k)
\newcommand{\lpdfxyk}[1][k]{\log p(\xv | y= #1)} % log p(x | y = k)
\newcommand{\pdfxiyk}[1][k]{p\left(\xi | y= #1 \right)} % p(x^i | y = k)

% prior probabilities
\newcommand{\pik}[1][k]{\pi_{#1}} % pi_k, prior
\newcommand{\lpik}[1][k]{\log \pi_{#1}} % log pi_k, log of the prior
\newcommand{\pit}{\pi(\thetav)} % Prior probability of parameter theta

% posterior probabilities
\newcommand{\post}{\P(y = 1 ~|~ \xv)} % P(y = 1 | x), post. prob for y=1
\newcommand{\postk}[1][k]{\P(y = #1 ~|~ \xv)} % P(y = k | y), post. prob for y=k
\newcommand{\pidomains}{\pi: \Xspace \rightarrow \unitint} % pi with domain and co-domain
\newcommand{\pibayes}{\pi^{\ast}} % Bayes-optimal classification model
\newcommand{\pixbayes}{\pi^{\ast}(\xv)} % Bayes-optimal classification model
\newcommand{\pix}{\pi(\xv)} % pi(x), P(y = 1 | x)
\newcommand{\piv}{\bm{\pi}} % pi, bold, as vector
\newcommand{\pikx}[1][k]{\pi_{#1}(\xv)} % pi_k(x), P(y = k | x)
\newcommand{\pikxt}[1][k]{\pi_{#1}(\xv ~|~ \thetav)} % pi_k(x | theta), P(y = k | x, theta)
\newcommand{\pixh}{\hat \pi(\xv)} % pi(x) hat, P(y = 1 | x) hat
\newcommand{\pikxh}[1][k]{\hat \pi_{#1}(\xv)} % pi_k(x) hat, P(y = k | x) hat
\newcommand{\pixih}{\hat \pi(\xi)} % pi(x^(i)) with hat
\newcommand{\pikxih}[1][k]{\hat \pi_{#1}(\xi)} % pi_k(x^(i)) with hat
\newcommand{\pdfygxt}{p(y ~|~\xv, \thetav)} % p(y | x, theta)
\newcommand{\pdfyigxit}{p\left(\yi ~|~\xi, \thetav\right)} % p(y^i |x^i, theta)
\newcommand{\lpdfygxt}{\log \pdfygxt } % log p(y | x, theta)
\newcommand{\lpdfyigxit}{\log \pdfyigxit} % log p(y^i |x^i, theta)

% probababilistic
\newcommand{\bayesrulek}[1][k]{\frac{\P(\xv | y= #1) \P(y= #1)}{\P(\xv)}} % Bayes rule
\newcommand{\muk}{\bm{\mu_k}} % mean vector of class-k Gaussian (discr analysis)

% residual and margin
\newcommand{\eps}{\epsilon} % residual, stochastic
\newcommand{\epsv}{\bm{\epsilon}} % residual, stochastic, as vector
\newcommand{\epsi}{\epsilon^{(i)}} % epsilon^i, residual, stochastic
\newcommand{\epsh}{\hat{\epsilon}} % residual, estimated
\newcommand{\epsvh}{\hat{\epsv}} % residual, estimated, vector
\newcommand{\yf}{y \fx} % y f(x), margin
\newcommand{\yfi}{\yi \fxi} % y^i f(x^i), margin
\newcommand{\Sigmah}{\hat \Sigma} % estimated covariance matrix
\newcommand{\Sigmahj}{\hat \Sigma_j} % estimated covariance matrix for the j-th class

% ml - loss, risk, likelihood
\newcommand{\Lyf}{L\left(y, f\right)} % L(y, f), loss function
\newcommand{\Lypi}{L\left(y, \pi\right)} % L(y, pi), loss function
\newcommand{\Lxy}{L\left(y, \fx\right)} % L(y, f(x)), loss function
\newcommand{\Lxyi}{L\left(\yi, \fxi\right)} % loss of observation
\newcommand{\Lxyt}{L\left(y, \fxt\right)} % loss with f parameterized
\newcommand{\Lxyit}{L\left(\yi, \fxit\right)} % loss of observation with f parameterized
\newcommand{\Lxym}{L\left(\yi, f\left(\bm{\tilde{x}}^{(i)} ~|~ \thetav\right)\right)} % loss of observation with f parameterized
\newcommand{\Lpixy}{L\left(y, \pix\right)} % loss in classification
\newcommand{\Lpiv}{L\left(y, \piv\right)} % loss in classification
\newcommand{\Lpixyi}{L\left(\yi, \pixii\right)} % loss of observation in classification
\newcommand{\Lpixyt}{L\left(y, \pixt\right)} % loss with pi parameterized
\newcommand{\Lpixyit}{L\left(\yi, \pixit\right)} % loss of observation with pi parameterized
\newcommand{\Lhxy}{L\left(y, \hx\right)} % L(y, h(x)), loss function on discrete classes
\newcommand{\Lr}{L\left(r\right)} % L(r), loss defined on residual (reg) / margin (classif)
\newcommand{\lone}{|y - \fx|} % L1 loss
\newcommand{\ltwo}{\left(y - \fx\right)^2} % L2 loss
\newcommand{\lbernoullimp}{\ln(1 + \exp(-y \cdot \fx))} % Bernoulli loss for -1, +1 encoding
\newcommand{\lbernoullizo}{- y \cdot \fx + \log(1 + \exp(\fx))} % Bernoulli loss for 0, 1 encoding
\newcommand{\lcrossent}{- y \log \left(\pix\right) - (1 - y) \log \left(1 - \pix\right)} % cross-entropy loss
\newcommand{\lbrier}{\left(\pix - y \right)^2} % Brier score
\newcommand{\risk}{\mathcal{R}} % R, risk
\newcommand{\riskbayes}{\mathcal{R}^\ast}
\newcommand{\riskf}{\risk(f)} % R(f), risk
\newcommand{\riskdef}{\E_{y|\xv}\left(\Lxy \right)} % risk def (expected loss)
\newcommand{\riskt}{\mathcal{R}(\thetav)} % R(theta), risk
\newcommand{\riske}{\mathcal{R}_{\text{emp}}} % R_emp, empirical risk w/o factor 1 / n
\newcommand{\riskeb}{\bar{\mathcal{R}}_{\text{emp}}} % R_emp, empirical risk w/ factor 1 / n
\newcommand{\riskef}{\riske(f)} % R_emp(f)
\newcommand{\risket}{\mathcal{R}_{\text{emp}}(\thetav)} % R_emp(theta)
\newcommand{\riskr}{\mathcal{R}_{\text{reg}}} % R_reg, regularized risk
\newcommand{\riskrt}{\mathcal{R}_{\text{reg}}(\thetav)} % R_reg(theta)
\newcommand{\riskrf}{\riskr(f)} % R_reg(f)
\newcommand{\riskrth}{\hat{\mathcal{R}}_{\text{reg}}(\thetav)} % hat R_reg(theta)
\newcommand{\risketh}{\hat{\mathcal{R}}_{\text{emp}}(\thetav)} % hat R_emp(theta)
\newcommand{\LL}{\mathcal{L}} % L, likelihood
\newcommand{\LLt}{\mathcal{L}(\thetav)} % L(theta), likelihood
\newcommand{\LLtx}{\mathcal{L}(\thetav | \xv)} % L(theta|x), likelihood
\newcommand{\logl}{\ell} % l, log-likelihood
\newcommand{\loglt}{\logl(\thetav)} % l(theta), log-likelihood
\newcommand{\logltx}{\logl(\thetav | \xv)} % l(theta|x), log-likelihood
\newcommand{\errtrain}{\text{err}_{\text{train}}} % training error
\newcommand{\errtest}{\text{err}_{\text{test}}} % test error
\newcommand{\errexp}{\overline{\text{err}_{\text{test}}}} % avg training error

% lm
\newcommand{\thx}{\thetav^\top \xv} % linear model
\newcommand{\olsest}{(\Xmat^\top \Xmat)^{-1} \Xmat^\top \yv} % OLS estimator in LM

% resampling
\newcommand{\ntest}{n_{\mathrm{test}}} % size of the test set
\newcommand{\ntrain}{n_{\mathrm{train}}} % size of the train set
\newcommand{\ntesti}[1][i]{n_{\mathrm{test},#1}} % size of the i-th test set
\newcommand{\ntraini}[1][i]{n_{\mathrm{train},#1}} % size of the i-th train set
\newcommand{\Jtrain}{J_\mathrm{train}} % index vector train data
\newcommand{\Jtest}{J_\mathrm{test}} % index vector test data
\newcommand{\Jtraini}[1][i]{J_{\mathrm{train},#1}} % index vector i-th train dataset
\newcommand{\Jtesti}[1][i]{J_{\mathrm{test},#1}} % index vector i-th test dataset
\newcommand{\Dtraini}[1][i]{\mathcal{D}_{\text{train},#1}} % D_train,i, i-th training set
\newcommand{\Dtesti}[1][i]{\mathcal{D}_{\text{test},#1}} % D_test,i, i-th test set

\newcommand{\JSpace}[1][m]{\nset^{#1}} % space of train indices of size n_train
\newcommand{\JtrainSpace}{\nset^{\ntrain}} % space of train indices of size n_train
\newcommand{\JtestSpace}{\nset^{\ntest}} % space of train indices of size n_test
\newcommand{\yJ}[1][J]{\yv_{#1}} % output vector associated to index J
\newcommand{\yJDef}{\left(y^{(J^{(1)})},\dots,y^{(J^{(m)})}\right)} % def of the output vector associated to index J
\newcommand{\JJ}{\mathcal{J}} % cali-J, set of all splits
\newcommand{\JJset}{\left((\Jtraini[1], \Jtesti[1]),\dots,(\Jtraini[B], \Jtesti[B])\right)} % (Jtrain_1,Jtest_1) ...(Jtrain_B,Jtest_B)
\newcommand{\Itrainlam}{\ind(\Dtrain, \lamv)}
% Generalization error
\newcommand{\GE}{\mathrm{GE}} % GE
\newcommand{\GEh}{\widehat{\GE}} % GE-hat
\newcommand{\GEfull}[1][\ntrain]{\GE(\ind, \lamv, #1, \rho)} % GE full
\newcommand{\GEhholdout}{\GEh_{\Jtrain, \Jtest}(\ind, \lamv, |\Jtrain|, \rho)} % GE hat holdout
\newcommand{\GEhholdouti}[1][i]{\GEh_{\Jtraini[#1], \Jtesti[#1]}(\ind, \lamv, |\Jtraini[#1]|, \rho)} % GE hat holdout i-th set
\newcommand{\GEhlam}{\GEh(\lamv)} % GE-hat(lam)
\newcommand{\GEhlamsubIJrho}{\GEh_{\ind, \JJ, \rho}(\lamv)} % GE-hat_I,J,rho(lam)
\newcommand{\GEhresa}{\GEh(\ind, \JJ, \rho, \lamv)} % GE-hat_I,J,rho(lam)
\newcommand{\GErhoDef}{\lim_{\ntest\rightarrow\infty} \E_{\Dtrain,\Dtest \sim \Pxy} \left[ \rho\left(\yv_{\Jtest}, \FJtestftrain\right)\right]} % GE formal def
\newcommand{\agr}{\mathrm{agr}} % aggregate function
\newcommand{\GEf}{\GE\left(\fh\right)} % GE of a fitted model
\newcommand{\GEfh}{\GEh\left(\fh\right)} % GEh of a fitted model
\newcommand{\GEfL}{\GE\left(\fh, L\right)} % GE of a fitted model wrt loss L
\newcommand{\Lyfhx}{L\left(y, \hat{f}(\xv)\right)} % pointwise loss of fitted model
\newcommand{\GEnf}[1]{GE_n\left(\fh_{#1}\right)} % GE of a fitted model
\newcommand{\GEind}{GE_n\left(\ind_{L, O}\right)} % GE of inducer
\newcommand{\GED}{\GE_{\D}} % GE indexed with data
\newcommand{\EGEn}{EGE_n} % expected GE
\newcommand{\EDn}{\E_{|D| = n}} % expectation wrt data of size n

% performance measure
\newcommand{\rhoL}{\rho_L} % perf. measure derived from pointwise loss
\newcommand{\F}{\bm{F}} % matrix of prediction scores
\newcommand{\Fi}[1][i]{\F^{(#1)}} % i-th row vector of the predscore mat
\newcommand{\FJ}[1][J]{\F_{#1}} % predscore mat idxvec J
\newcommand{\FJf}{\FJ[J,f]} % predscore mat idxvec J and model f
\newcommand{\FJtestfh}{\FJ[\Jtest, \fh]} % predscore mat idxvec Jtest and model f hat
\newcommand{\FJtestftrain}{\F_{\Jtest, \Itrainlam}} % predscore mat idxvec Jtest and model f
\newcommand{\FJtestftraini}[1][i]{\F_{\Jtesti[#1],\ind(\Dtraini[#1], \lamv)}}  % predscore mat i-th idxvec Jtest and model f
\newcommand{\FJfDef}{\left(f(\xv^{(J^{(1)})}),\dots, f(\xv^{(J^{(m)})})\right)} % def of predscore mat idxvec J and model f
\newcommand{\preimageRho}{\bigcup_{m\in\N}\left(\Yspace^m\times\R^{m\times g}\right)} % Set of all datasets times HP space

% ml - ROC
\newcommand{\np}{n_{+}} % no. of positive instances
\newcommand{\nn}{n_{-}} % no. of negative instances
\newcommand{\rn}{\pi_{-}} % proportion negative instances
\newcommand{\rp}{\pi_{+}} % proportion negative instances
  % true/false pos/neg:
\newcommand{\tp}{\# \text{TP}} % true pos
\newcommand{\fap}{\# \text{FP}} % false pos (fp taken for partial derivs)
\newcommand{\tn}{\# \text{TN}} % true neg
\newcommand{\fan}{\# \text{FN}} % false neg

\input{../../latex-math/ml-hpo.tex}

%%

\newcommand{\titlefigure}{figure_man/roc_metrics}
\newcommand{\learninggoals}{
\item Understand why accuracy is not an optimal performance measure for 
imbalanced labels
\item Understand the different measures computable from a confusion matrix
\item Be aware that each of these measures has a variety of names}


\title{Introduction to Machine Learning}
% \author{Bernd Bischl, Christoph Molnar, Daniel Schalk, Fabian Scheipl}
\institute{\href{https://compstat-lmu.github.io/lecture_i2ml/}{compstat-lmu.github.io/lecture\_i2ml}}
\date{}

\begin{document}

\lecturechapter{Evaluation: ROC Basics}
\lecture{Introduction to Machine Learning}
\sloppy

% ------------------------------------------------------------------------------

% \begin{vbframe}{Imbalanced Binary Labels}
% 
% \begin{center}
% % FIGURE SOURCE: https://docs.google.com/drawings/d/1WERS9WXwS4zla86fk6ESQkskNN1WZMI1YCPprnp0Ew0/edit?usp=sharing
% \includegraphics[width=.9\textwidth]{figure_man/imbalanced.pdf}\\
% Classify all as \enquote{no disease} (green) $\rightarrow$ high accuracy.
% 
% \lz
% 
% \textbf{Accuracy Paradox}
% \end{center}
% 
% \end{vbframe}

% ------------------------------------------------------------------------------

% \begin{vbframe}{Imbalanced Costs}
% 
% \begin{center}
% % FIGURE SOURCE: https://docs.google.com/drawings/d/1GlmMqzpeNHU_rtPFIrJMlY9Iz6XexvHEwTl3dNYKyQU/edit?usp=sharing
% \includegraphics[width=.3\textwidth]{figure_man/imbalanced-costs.pdf}\\
% Classify incorrectly as \enquote{no disease} $\rightarrow$ very high cost
% 
% \end{center}
% 
% \end{vbframe}

% ------------------------------------------------------------------------------

\begin{vbframe}{class imbalance}

\begin{itemize}
 \item Assume a binary classifier diagnoses a serious medical 
 condition.
 \item Label distribution is often \textbf{imbalanced}, i.e, not many 
 people have the disease.
 \item Evaluating on mce is often inappropriate for scenarios with 
 imbalanced labels:
 \begin{itemize}
   \item Assume that only 0.5\,\% have the disease.
   \item Always predicting \enquote{no disease} has an mce of 0.5\,\%, 
   corresponding to very high accuracy.
   \item This sends all sick patients home $\rightarrow$ bad system % -- even classifying everyone as \enquote{disease} might be 
%   better (depending on the treatment).
 \end{itemize}
 \item This problem is known as the \textbf{accuracy paradox}.
\end{itemize}

\framebreak

Classifying all observations as \enquote{no disease} (green) yields top 
accuracy simply because the \enquote{disease} occurs so rarely 
$\rightarrow$ accuracy paradox.

\lz

\begin{center}
  % FIGURE SOURCE: https://docs.google.com/drawings/d  /1WERS9WXwS4zla86fk6ESQkskNN1WZMI1YCPprnp0Ew0/edit?usp=sharing
  \includegraphics[width=0.7\textwidth]{figure_man/imbalanced.pdf}
\end{center}

\end{vbframe}
 
% ------------------------------------------------------------------------------

\begin{vbframe}{imbalanced costs}
 
\begin{itemize}
  \item Another point of view is \textbf{imbalanced costs}.
  \item In our example, classifying a sick patient as healthy should incur a 
  much higher cost than classifying a healthy patient as sick.
  \item The costs depend a lot on what happens next: we can well assume that 
  our system is some type of screening filter, and often the next step after 
  labeling someone as sick might be a more invasive, expensive, but also  more 
  reliable test for the disease.
  \item Erroneously subjecting someone to this step is undesirable 
  (psychological, economic, medical expense), but sending someone home to get 
  worse or die seems much more so.
  \item Such situations not only arise under label imbalance, but also when 
  costs differ (even though classes might be balanced).
  \item We could see this as imbalanced costs of misclassification, rather than 
  imbalanced labels; both situations are tightly connected.
\end{itemize}

\framebreak

\lz

\begin{minipage}[c]{0.65\textwidth}
  \raggedright
  \textbf{Imbalanced costs: } classifying incorrectly as \enquote{no disease} 
  incurs very high cost.
\end{minipage}%
\begin{minipage}[c]{0.35\textwidth}
  \centering
  \includegraphics[trim = 0 0 0 10, clip, width=0.4\textwidth]
  {figure_man/imbalanced-costs.pdf}
\end{minipage}

\lz

\begin{itemize}
  \item Problem: if we were able to specify costs precisely, we could evaluate 
  or even optimize on them.
  \item This important subfield of ML is called \textbf{cost-sensitive 
  learning}, which we will not cover in this lecture unit.
  \item Unfortunately, users find it notoriously hard to come up with 
  precise cost figures in imbalanced scenarios.
  \item Evaluating \enquote{from different perspectives}, with multiple metrics, 
  often helps to get a first impression of system quality.
\end{itemize}
 
\end{vbframe}
 
% ------------------------------------------------------------------------------
 
 % \begin{vbframe}{Binary Classifiers and Costs}
 % \begin{itemize}
 %   \item Problem is: If we could specify costs precisely, we could evaluate against them, we might even optimize our model for them
 %   \item This important subfield of ML is called \textbf{cost-sensitive learning}, which we will not cover in this lecture unit
 %   \item Unfortunately, users often have a notoriously hard time to come up with precise cost numbers in imbalanced scenarios
 %   \item Evaluating "from different perspectives", with multiple metrics, often helps, especially to get a first impression
 %     of the quality of the system
 % \end{itemize}
 % \end{vbframe}

% ------------------------------------------------------------------------------

\begin{frame}{ROC Analysis}

\begin{itemize}
  \item \textbf{ROC analysis} is a subfield of ML which studies the evaluation 
  of binary prediction systems.
  \item ROC stands for \enquote{receiver operating characteristics} and was 
  initially developed by electrical engineers and radar engineers during World 
  War II for detecting enemy objects in battlefields -- still has the funny 
  name.
\end{itemize}

\lz

\begin{center}
\includegraphics[width=.4\textwidth]{figure_man/receiver_operator.jpg}
{\tiny \url{http://media.iwm.org.uk/iwm/mediaLib//39/media-39665/large.jpg}}
\end{center}

\end{frame}

% ------------------------------------------------------------------------------

%  \begin{vbframe}{Confusion Matrix and ROC Metrics}
%  \begin{itemize}
%    \item From now on, we will call one class "positive", one "negative" and 
%    their respective sizes $n_+$ and $n_-$.
%    \item The positive class is the more important, often smaller one.
%    \item We represent all predictions in a confusion matrix and count correct 
%    and incorrect class assignments
%    \item False Positive means: We assigned "positive", but were wrong
%  \end{itemize}
% % % FIGURE SOURCE: No source
%  \includegraphics[width=0.7\textwidth]{figure_man/roc-confmatrix1.png}
%  \end{vbframe}

% ------------------------------------------------------------------------------

% \begin{vbframe}{Confusion Matrix}
% 
% \begin{center}
% \small
% \renewcommand{\arraystretch}{1.5}
% \begin{tabular}{cc||cc}
%     & & \multicolumn{2}{c}{\bfseries True Class $y$}  \\
%     & & $+$ & $-$  \\ 
%     \hline \hline
%     \bfseries Pred.     & $+$ & TP & FP\\
%               $\yh$ & $-$ & FN & TN\\ 
% \end{tabular}
% \renewcommand{\arraystretch}{1}
% \end{center}
% 
% \begin{itemize}
%   \item $+$: \enquote{positive} class
%   \item $-$: \enquote{negative} class
%   \item $\np$: number of observations in $+$
%   \item $\nn$: number of observations in $-$
% \end{itemize}
% \end{vbframe}

% ------------------------------------------------------------------------------

\begin{vbframe}{Labels: ROC Metrics}
From the confusion matrix (binary case), we can calculate "ROC" metrics.

% % FIGURE SOURCE: No source
% \includegraphics[width=0.7\textwidth]{figure_man/roc-confmatrix2.png}

% \begin{center}
% \small
% \begin{tabular}{cc|>{\centering\arraybackslash}p{7em}>{\centering\arraybackslash}p{8em}|>{\centering\arraybackslash}p{8em}}
%     & & \multicolumn{2}{c}{\bfseries True Class $y$} & \\
%     & & $+$ & $-$ & \\
%     \hline
%     \bfseries Pred.     & $+$ & True Positive (TP)  & False Positive (FP) & Positive Predictive Value (PPV) = $\frac{\text{TP}}{\text{TP} + \text{FP}}$\\
%               $\hat{y}$ & $-$ & False Negative (FN) & True Negative (TN) & Negative Predictive Value (NPV) = $\frac{\text{TN}}{\text{FN} + \text{TN}}$\\
%     \hline
%     & & TPR = $\frac{\text{TP}}{\text{TP} + \text{FN}}$ & TNR = $\frac{\text{TN}}{\text{FP} + \text{TN}}$ & Accuracy = $\frac{\text{TP}+ \text{TN}}{\text{TOTAL}}$
% \end{tabular}
% \end{center}

\begin{center}
\small
\renewcommand{\arraystretch}{1.5}
\begin{tabular}{cc||cc|c}
    & & \multicolumn{2}{c|}{\bfseries True Class $y$} & \\
    & & $+$ & $-$ & \\ 
    \hline \hline
    \bfseries Pred.     & $+$ & TP & FP & $\rho_{PPV} = \frac{\text{TP}}{\text{TP} + \text{FP}}$\\
              $\yh$ & $-$ & FN & TN & $\rho_{NPV} = \frac{\text{TN}}{\text{FN} + \text{TN}}$\\
    \hline
    & & $\rho_{TPR} = \frac{\text{TP}}{\text{TP} + \text{FN}}$ & $\rho_{TNR} = \frac{\text{TN}}{\text{FP} + \text{TN}}$ & $\rho_{ACC} = \frac{\text{TP}+ \text{TN}}{\text{TOTAL}}$
\end{tabular}
\renewcommand{\arraystretch}{1}
\end{center}

\begin{itemize}
  \small
  \item True positive rate $\rho_{TPR}$: how many of the true 1s did we predict 
  as 1?
  \item True Negative rate $\rho_{TNR}$: how many of the true 0s did we predict 
  as 0?
  \item Positive predictive value $\rho_{PPV}$: if we predict 1, how likely is 
  it a true 1?
  \item Negative predictive value $\rho_{NPV}$: if we predict 0, how likely is 
  it a true 0?
  \item Accuracy $\rho_{ACC}$: how many instances did we predict correctly?
\end{itemize}
\end{vbframe}

% ------------------------------------------------------------------------------

\begin{vbframe}{Labels: ROC Metrics}

Example:

\begin{center}
  % FIGURE SOURCE: No source
  \includegraphics[width=\textwidth]{figure_man/roc-confmatrix-example.png}
  \tiny \url{https://en.wikipedia.org/wiki/Receiver_operating_characteristic}
\end{center}

\end{vbframe}

% ------------------------------------------------------------------------------

\begin{vbframe}{More metrics and alternative terminology}

Unfortunately, for many concepts in ROC, 2-3 different terms exist.

\begin{center}
% FIGURE SOURCE: https://en.wikipedia.org/wiki/F1_score#Diagnostic_testing
\includegraphics[width=0.95\textwidth]{figure_man/roc-confmatrix-allterms.png}
\end{center}
\href{https://en.wikipedia.org/wiki/F1_score#Diagnostic_testing}{\beamergotobutton{Clickable version/picture source}} $\phantom{blablabla}$
\href{https://upload.wikimedia.org/wikipedia/commons/0/0e/DiagnosticTesting_Diagram.svg}{\beamergotobutton{Interactive diagram}}
\end{vbframe}

% ------------------------------------------------------------------------------

\begin{vbframe}{Labels: $F_1$ Measure}

\small

\begin{itemize}
  \item It is difficult to achieve high \textbf{positive predictive value} and 
  high \textbf{true positive rate} simultaneously.
   \item A classifier predicting more positive will be more 
   sensitive (higher $\rho_{TPR}$), but it will also tend to give more 
   \textit{false} positives (lower $\rho_{TNR}$, lower $\rho_{PPV}$).
   \item A classifier that predicts more negatives will be more precise 
   (higher $\rho_{PPV}$), but it will also produce more \textit{false} negatives 
   (lower $\rho_{TPR}$).
 \end{itemize}

The \textbf{$F_1$ score} balances two conflicting goals:\\%[.5em]
\begin{enumerate}
 \item Maximizing positive predictive value
 \item Maximizing true positive rate \\%[.5em]
\end{enumerate}

$\rho_{F_1}$ is the harmonic mean of $\rho_{PPV}$ and $\rho_{TPR}$:
$$\rho_{F_1} = 2 \cdot \cfrac{\rho_{PPV} \cdot \rho_{TPR}}{\rho_{PPV} + 
\rho_{TPR}}$$

Note that this measure still does not account for the number of true
negatives.

\framebreak

\normalsize

\begin{minipage}[c]{0.5\textwidth}
  \small
  $F_1$ score for different combinations of $\rho_{PPV}$ \& $\rho_{TPR}$. \\
  $\rightarrow$ Tends more towards the lower of the two combined values.
\end{minipage}%
\begin{minipage}[c]{0.5\textwidth}
  \centering
  \includegraphics[width=0.8\textwidth]{figure/eval_mclass_roc.pdf}
\end{minipage}

% Tabulated $F_1$-Score for different TPR (rows) and PPV (cols) combinations:
% \begin{knitrout}\scriptsize
% \definecolor{shadecolor}{rgb}{0.969, 0.969, 0.969}\color{fgcolor}
% 
% \includegraphics[width=0.5\textwidth]{figure/eval_mclass_roc.pdf}
% 
% \end{knitrout}
% $\rightarrow$ Tends more towards the lower of the two combined values.

\begin{itemize}
  \item A model with $\rho_{TPR} = 0$ (no positive instance predicted as 
  positive) or 
  $\rho_{PPV} = 0$ (no true positives among the predicted) has $\rho_{F_1} = 0$.
  \item Always predicting \enquote{negative}: $\rho_{F_1} = 0$.
  \item Always predicting \enquote{positive}: $\rho_{F_1} = 2 \cdot \rho_{PPV} / 
  (\rho_{PPV} + 1) = 2 \cdot \np / (\np + n)$,\\ 
  which will be small when the size of the positive class $\np$ is small.
\end{itemize}

\end{vbframe}

% ------------------------------------------------------------------------------

\begin{vbframe}{which metric to use?}

\begin{itemize}
  \footnotesize
  \item As we have seen, there is a plethora of methods. \\
  $\rightarrow$ This leaves practitioners with the question of which to use.
  \item Consider a small benchmark study.
  \begin{itemize}
    \footnotesize
    \item We let $k$-NN, logistic regression, a classification tree, and a random 
    forest compete on classifying the \texttt{credit risk} data.
    \item The data consist of 1000 observations of borrowers' financial 
    situation and their creditworthiness (good/bad) as target.
    \item Predicted probabilities are thresholded at 0.5 for the positive class.
    % \item We benchmark our learners on accuracy, AUC, $F_1$, precision, recall 
    % and specificity.
    \item Depending on the metric we use, learners are ranked differently 
    according to performance (value of respective performance measure in 
    parentheses):
  \end{itemize}
\end{itemize}

\begin{center}
\includegraphics[width=0.55\textwidth]{figure/eval_mclass_benchmark.pdf}
\end{center}

\framebreak

\begin{itemize}
  \item We need not expect overly large discrepancies in general, but neither 
  will we always see an unambiguous picture. 
  \item Different metrics emphasize different aspects of performance. \\
  $\rightarrow$ The choice should be made in the domain context.
  \item For practitioners it is vital to understand what should be 
  evaluated exactly, and which measure is appropriate.  
  \begin{itemize}
    \item Regarding credit risk, for instance, defaults are to be avoided, but 
    not at all cost.
    \item The bank must undertake a certain risk to remain profitable, so a more 
    balanced measure such as the $F_1$ score might be in order.
    \item On the other hand, a system detecting weapons at an airport should be 
    able to achieve very high true positive rates, even if this comes at the 
    expense of some false alarms.
  \end{itemize}
\end{itemize}

\end{vbframe}

% ------------------------------------------------------------------------------

\endlecture

\end{document}
