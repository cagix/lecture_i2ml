\documentclass[11pt,compress,t,notes=noshow, xcolor=table]{beamer}

\usepackage[]{graphicx}\usepackage[]{color}
% maxwidth is the original width if it is less than linewidth
% otherwise use linewidth (to make sure the graphics do not exceed the margin)
\makeatletter
\def\maxwidth{ %
  \ifdim\Gin@nat@width>\linewidth
    \linewidth
  \else
    \Gin@nat@width
  \fi
}
\makeatother



\definecolor{fgcolor}{rgb}{0.345, 0.345, 0.345}
\newcommand{\hlnum}[1]{\textcolor[rgb]{0.686,0.059,0.569}{#1}}%
\newcommand{\hlstr}[1]{\textcolor[rgb]{0.192,0.494,0.8}{#1}}%
\newcommand{\hlcom}[1]{\textcolor[rgb]{0.678,0.584,0.686}{\textit{#1}}}%
\newcommand{\hlopt}[1]{\textcolor[rgb]{0,0,0}{#1}}%
\newcommand{\hlstd}[1]{\textcolor[rgb]{0.345,0.345,0.345}{#1}}%
\newcommand{\hlkwa}[1]{\textcolor[rgb]{0.161,0.373,0.58}{\textbf{#1}}}%
\newcommand{\hlkwb}[1]{\textcolor[rgb]{0.69,0.353,0.396}{#1}}%
\newcommand{\hlkwc}[1]{\textcolor[rgb]{0.333,0.667,0.333}{#1}}%
\newcommand{\hlkwd}[1]{\textcolor[rgb]{0.737,0.353,0.396}{\textbf{#1}}}%
\let\hlipl\hlkwb

\usepackage{framed}
\makeatletter
\newenvironment{kframe}{%
 \def\at@end@of@kframe{}%
 \ifinner\ifhmode%
  \def\at@end@of@kframe{\end{minipage}}%
  \begin{minipage}{\columnwidth}%
 \fi\fi%
 \def\FrameCommand##1{\hskip\@totalleftmargin \hskip-\fboxsep
 \colorbox{shadecolor}{##1}\hskip-\fboxsep
     % There is no \\@totalrightmargin, so:
     \hskip-\linewidth \hskip-\@totalleftmargin \hskip\columnwidth}%
 \MakeFramed {\advance\hsize-\width
   \@totalleftmargin\z@ \linewidth\hsize
   \@setminipage}}%
 {\par\unskip\endMakeFramed%
 \at@end@of@kframe}
\makeatother

\definecolor{shadecolor}{rgb}{.97, .97, .97}
\definecolor{messagecolor}{rgb}{0, 0, 0}
\definecolor{warningcolor}{rgb}{1, 0, 1}
\definecolor{errorcolor}{rgb}{1, 0, 0}
\newenvironment{knitrout}{}{} % an empty environment to be redefined in TeX

\usepackage{alltt}
\newcommand{\SweaveOpts}[1]{}  % do not interfere with LaTeX
\newcommand{\SweaveInput}[1]{} % because they are not real TeX commands
\newcommand{\Sexpr}[1]{}       % will only be parsed by R
\newcommand{\xmark}{\ding{55}}%


\usepackage[english]{babel}
\usepackage[utf8]{inputenc}

\usepackage{dsfont}
\usepackage{verbatim}
\usepackage{amsmath}
\usepackage{amsfonts}
\usepackage{amssymb}
\usepackage{bm}
\usepackage{csquotes}
\usepackage{multirow}
\usepackage{longtable}
\usepackage{booktabs}
\usepackage{enumerate}
\usepackage[absolute,overlay]{textpos}
\usepackage{psfrag}
\usepackage{algorithm}
\usepackage{algpseudocode}
\usepackage{eqnarray}
\usepackage{arydshln}
\usepackage{tabularx}
\usepackage{placeins}
\usepackage{tikz}
\usepackage{setspace}
\usepackage{colortbl}
\usepackage{mathtools}
\usepackage{wrapfig}
\usepackage{bm}
\usepackage{amsmath}
\usepackage{pifont}

\usetikzlibrary{shapes,arrows,automata,positioning,calc,chains,trees, shadows}
\tikzset{
  %Define standard arrow tip
  >=stealth',
  %Define style for boxes
  punkt/.style={
    rectangle,
    rounded corners,
    draw=black, very thick,
    text width=6.5em,
    minimum height=2em,
    text centered},
  % Define arrow style
  pil/.style={
    ->,
    thick,
    shorten <=2pt,
    shorten >=2pt,}
}

\usepackage{subfig}

% Defines macros and environments
\usepackage{../../style/lmu-lecture}


\let\code=\texttt
\let\proglang=\textsf

\setkeys{Gin}{width=0.9\textwidth}

\setbeamertemplate{frametitle}{\expandafter\uppercase\expandafter\insertframetitle}

% This file is included in slides and exercises

% Rarely used fontstyle for R packages, used only in 
% - forests/slides-forests-benchmark.tex
% - exercises/single-exercises/methods_l_1.Rnw
% - slides/cart/attic/slides_extra_trees.Rnw
\newcommand{\pkg}[1]{{\fontseries{b}\selectfont #1}}

% Spacing helpers, used often (mostly in exercises for \dlz)
\newcommand{\lz}{\vspace{0.5cm}} % vertical space (used often in slides)
\newcommand{\dlz}{\vspace{1cm}}  % double vertical space (used often in exercises, never in slides)
\newcommand{\oneliner}[1] % Oneliner for important statements, used e.g. in iml, algods
{\begin{block}{}\begin{center}\begin{Large}#1\end{Large}\end{center}\end{block}}

% Don't know if this is used or needed, remove?
% textcolor that works in mathmode
% https://tex.stackexchange.com/a/261480
% Used e.g. in forests/slides-forests-bagging.tex
% [...] \textcolor{blue}{\tfrac{1}{M}\sum^M_{m} [...]
% \makeatletter
% \renewcommand*{\@textcolor}[3]{%
%   \protect\leavevmode
%   \begingroup
%     \color#1{#2}#3%
%   \endgroup
% }
% \makeatother





% dependencies: amsmath, amssymb, dsfont
% math spaces
\ifdefined\N
\renewcommand{\N}{\mathds{N}} % N, naturals
\else \newcommand{\N}{\mathds{N}} \fi
\newcommand{\Z}{\mathds{Z}} % Z, integers
\newcommand{\Q}{\mathds{Q}} % Q, rationals
\newcommand{\R}{\mathds{R}} % R, reals
\ifdefined\C
\renewcommand{\C}{\mathds{C}} % C, complex
\else \newcommand{\C}{\mathds{C}} \fi
\newcommand{\continuous}{\mathcal{C}} % C, space of continuous functions
\newcommand{\M}{\mathcal{M}} % machine numbers
\newcommand{\epsm}{\epsilon_m} % maximum error

% counting / finite sets
\newcommand{\setzo}{\{0, 1\}} % set 0, 1
\newcommand{\setmp}{\{-1, +1\}} % set -1, 1
\newcommand{\unitint}{[0, 1]} % unit interval

% basic math stuff
\newcommand{\xt}{\tilde x} % x tilde
\DeclareMathOperator*{\argmax}{arg\,max} % argmax
\DeclareMathOperator*{\argmin}{arg\,min} % argmin
\newcommand{\argminlim}{\mathop{\mathrm{arg\,min}}\limits} % argmax with limits
\newcommand{\argmaxlim}{\mathop{\mathrm{arg\,max}}\limits} % argmin with limits
\newcommand{\sign}{\operatorname{sign}} % sign, signum
\newcommand{\I}{\mathbb{I}} % I, indicator
\newcommand{\order}{\mathcal{O}} % O, order
\newcommand{\bigO}{\mathcal{O}} % Big-O Landau
\newcommand{\littleo}{{o}} % Little-o Landau
\newcommand{\pd}[2]{\frac{\partial{#1}}{\partial #2}} % partial derivative
\newcommand{\floorlr}[1]{\left\lfloor #1 \right\rfloor} % floor
\newcommand{\ceillr}[1]{\left\lceil #1 \right\rceil} % ceiling
\newcommand{\indep}{\perp \!\!\! \perp} % independence symbol

% sums and products
\newcommand{\sumin}{\sum\limits_{i=1}^n} % summation from i=1 to n
\newcommand{\sumim}{\sum\limits_{i=1}^m} % summation from i=1 to m
\newcommand{\sumjn}{\sum\limits_{j=1}^n} % summation from j=1 to p
\newcommand{\sumjp}{\sum\limits_{j=1}^p} % summation from j=1 to p
\newcommand{\sumik}{\sum\limits_{i=1}^k} % summation from i=1 to k
\newcommand{\sumkg}{\sum\limits_{k=1}^g} % summation from k=1 to g
\newcommand{\sumjg}{\sum\limits_{j=1}^g} % summation from j=1 to g
\newcommand{\summM}{\sum\limits_{m=1}^M} % summation from m=1 to M
\newcommand{\meanin}{\frac{1}{n} \sum\limits_{i=1}^n} % mean from i=1 to n
\newcommand{\meanim}{\frac{1}{m} \sum\limits_{i=1}^m} % mean from i=1 to n
\newcommand{\meankg}{\frac{1}{g} \sum\limits_{k=1}^g} % mean from k=1 to g
\newcommand{\meanmM}{\frac{1}{M} \sum\limits_{m=1}^M} % mean from m=1 to M
\newcommand{\prodin}{\prod\limits_{i=1}^n} % product from i=1 to n
\newcommand{\prodkg}{\prod\limits_{k=1}^g} % product from k=1 to g
\newcommand{\prodjp}{\prod\limits_{j=1}^p} % product from j=1 to p

% linear algebra
\newcommand{\one}{\bm{1}} % 1, unitvector
\newcommand{\zero}{\mathbf{0}} % 0-vector
\newcommand{\id}{\bm{I}} % I, identity
\newcommand{\diag}{\operatorname{diag}} % diag, diagonal
\newcommand{\trace}{\operatorname{tr}} % tr, trace
\newcommand{\spn}{\operatorname{span}} % span
\newcommand{\scp}[2]{\left\langle #1, #2 \right\rangle} % <.,.>, scalarproduct
\newcommand{\mat}[1]{\begin{pmatrix} #1 \end{pmatrix}} % short pmatrix command
\newcommand{\Amat}{\mathbf{A}} % matrix A
\newcommand{\Deltab}{\mathbf{\Delta}} % error term for vectors

% basic probability + stats
\renewcommand{\P}{\mathds{P}} % P, probability
\newcommand{\E}{\mathds{E}} % E, expectation
\newcommand{\var}{\mathsf{Var}} % Var, variance
\newcommand{\cov}{\mathsf{Cov}} % Cov, covariance
\newcommand{\corr}{\mathsf{Corr}} % Corr, correlation
\newcommand{\normal}{\mathcal{N}} % N of the normal distribution
\newcommand{\iid}{\overset{i.i.d}{\sim}} % dist with i.i.d superscript
\newcommand{\distas}[1]{\overset{#1}{\sim}} % ... is distributed as ...

% machine learning
\newcommand{\Xspace}{\mathcal{X}} % X, input space
\newcommand{\Yspace}{\mathcal{Y}} % Y, output space
\newcommand{\Zspace}{\mathcal{Z}} % Z, space of sampled datapoints
\newcommand{\nset}{\{1, \ldots, n\}} % set from 1 to n
\newcommand{\pset}{\{1, \ldots, p\}} % set from 1 to p
\newcommand{\gset}{\{1, \ldots, g\}} % set from 1 to g
\newcommand{\Pxy}{\mathbb{P}_{xy}} % P_xy
\newcommand{\Exy}{\mathbb{E}_{xy}} % E_xy: Expectation over random variables xy
\newcommand{\xv}{\mathbf{x}} % vector x (bold)
\newcommand{\xtil}{\tilde{\mathbf{x}}} % vector x-tilde (bold)
\newcommand{\yv}{\mathbf{y}} % vector y (bold)
\newcommand{\xy}{(\xv, y)} % observation (x, y)
\newcommand{\xvec}{\left(x_1, \ldots, x_p\right)^\top} % (x1, ..., xp)
\newcommand{\Xmat}{\mathbf{X}} % Design matrix
\newcommand{\allDatasets}{\mathds{D}} % The set of all datasets
\newcommand{\allDatasetsn}{\mathds{D}_n}  % The set of all datasets of size n
\newcommand{\D}{\mathcal{D}} % D, data
\newcommand{\Dn}{\D_n} % D_n, data of size n
\newcommand{\Dtrain}{\mathcal{D}_{\text{train}}} % D_train, training set
\newcommand{\Dtest}{\mathcal{D}_{\text{test}}} % D_test, test set
\newcommand{\xyi}[1][i]{\left(\xv^{(#1)}, y^{(#1)}\right)} % (x^i, y^i), i-th observation
\newcommand{\Dset}{\left( \xyi[1], \ldots, \xyi[n]\right)} % {(x1,y1)), ..., (xn,yn)}, data
\newcommand{\defAllDatasetsn}{(\Xspace \times \Yspace)^n} % Def. of the set of all datasets of size n
\newcommand{\defAllDatasets}{\bigcup_{n \in \N}(\Xspace \times \Yspace)^n} % Def. of the set of all datasets
\newcommand{\xdat}{\left\{ \xv^{(1)}, \ldots, \xv^{(n)}\right\}} % {x1, ..., xn}, input data
\newcommand{\ydat}{\left\{ \yv^{(1)}, \ldots, \yv^{(n)}\right\}} % {y1, ..., yn}, input data
\newcommand{\yvec}{\left(y^{(1)}, \hdots, y^{(n)}\right)^\top} % (y1, ..., yn), vector of outcomes
\newcommand{\greekxi}{\xi} % Greek letter xi
\renewcommand{\xi}[1][i]{\xv^{(#1)}} % x^i, i-th observed value of x
\newcommand{\yi}[1][i]{y^{(#1)}} % y^i, i-th observed value of y
\newcommand{\xivec}{\left(x^{(i)}_1, \ldots, x^{(i)}_p\right)^\top} % (x1^i, ..., xp^i), i-th observation vector
\newcommand{\xj}{\xv_j} % x_j, j-th feature
\newcommand{\xjvec}{\left(x^{(1)}_j, \ldots, x^{(n)}_j\right)^\top} % (x^1_j, ..., x^n_j), j-th feature vector
\newcommand{\phiv}{\mathbf{\phi}} % Basis transformation function phi
\newcommand{\phixi}{\mathbf{\phi}^{(i)}} % Basis transformation of xi: phi^i := phi(xi)

%%%%%% ml - models general
\newcommand{\lamv}{\bm{\lambda}} % lambda vector, hyperconfiguration vector
\newcommand{\Lam}{\bm{\Lambda}}	 % Lambda, space of all hpos
% Inducer / Inducing algorithm
\newcommand{\preimageInducer}{\left(\defAllDatasets\right)\times\Lam} % Set of all datasets times the hyperparameter space
\newcommand{\preimageInducerShort}{\allDatasets\times\Lam} % Set of all datasets times the hyperparameter space
% Inducer / Inducing algorithm
\newcommand{\ind}{\mathcal{I}} % Inducer, inducing algorithm, learning algorithm

% continuous prediction function f
\newcommand{\ftrue}{f_{\text{true}}}  % True underlying function (if a statistical model is assumed)
\newcommand{\ftruex}{\ftrue(\xv)} % True underlying function (if a statistical model is assumed)
\newcommand{\fx}{f(\xv)} % f(x), continuous prediction function
\newcommand{\fdomains}{f: \Xspace \rightarrow \R^g} % f with domain and co-domain
\newcommand{\Hspace}{\mathcal{H}} % hypothesis space where f is from
\newcommand{\fbayes}{f^{\ast}} % Bayes-optimal model
\newcommand{\fxbayes}{f^{\ast}(\xv)} % Bayes-optimal model
\newcommand{\fkx}[1][k]{f_{#1}(\xv)} % f_j(x), discriminant component function
\newcommand{\fh}{\hat{f}} % f hat, estimated prediction function
\newcommand{\fxh}{\fh(\xv)} % fhat(x)
\newcommand{\fxt}{f(\xv ~|~ \thetav)} % f(x | theta)
\newcommand{\fxi}{f\left(\xv^{(i)}\right)} % f(x^(i))
\newcommand{\fxih}{\hat{f}\left(\xv^{(i)}\right)} % f(x^(i))
\newcommand{\fxit}{f\left(\xv^{(i)} ~|~ \thetav\right)} % f(x^(i) | theta)
\newcommand{\fhD}{\fh_{\D}} % fhat_D, estimate of f based on D
\newcommand{\fhDtrain}{\fh_{\Dtrain}} % fhat_Dtrain, estimate of f based on D
\newcommand{\fhDnlam}{\fh_{\Dn, \lamv}} %model learned on Dn with hp lambda
\newcommand{\fhDlam}{\fh_{\D, \lamv}} %model learned on D with hp lambda
\newcommand{\fhDnlams}{\fh_{\Dn, \lamv^\ast}} %model learned on Dn with optimal hp lambda
\newcommand{\fhDlams}{\fh_{\D, \lamv^\ast}} %model learned on D with optimal hp lambda

% discrete prediction function h
\newcommand{\hx}{h(\xv)} % h(x), discrete prediction function
\newcommand{\hh}{\hat{h}} % h hat
\newcommand{\hxh}{\hat{h}(\xv)} % hhat(x)
\newcommand{\hxt}{h(\xv | \thetav)} % h(x | theta)
\newcommand{\hxi}{h\left(\xi\right)} % h(x^(i))
\newcommand{\hxit}{h\left(\xi ~|~ \thetav\right)} % h(x^(i) | theta)
\newcommand{\hbayes}{h^{\ast}} % Bayes-optimal classification model
\newcommand{\hxbayes}{h^{\ast}(\xv)} % Bayes-optimal classification model

% yhat
\newcommand{\yh}{\hat{y}} % yhat for prediction of target
\newcommand{\yih}{\hat{y}^{(i)}} % yhat^(i) for prediction of ith targiet
\newcommand{\resi}{\yi- \yih}

% theta
\newcommand{\thetah}{\hat{\theta}} % theta hat
\newcommand{\thetav}{\bm{\theta}} % theta vector
\newcommand{\thetavh}{\bm{\hat\theta}} % theta vector hat
\newcommand{\thetat}[1][t]{\thetav^{[#1]}} % theta^[t] in optimization
\newcommand{\thetatn}[1][t]{\thetav^{[#1 +1]}} % theta^[t+1] in optimization
\newcommand{\thetahDnlam}{\thetavh_{\Dn, \lamv}} %theta learned on Dn with hp lambda
\newcommand{\thetahDlam}{\thetavh_{\D, \lamv}} %theta learned on D with hp lambda
\newcommand{\mint}{\min_{\thetav \in \Theta}} % min problem theta
\newcommand{\argmint}{\argmin_{\thetav \in \Theta}} % argmin theta

% densities + probabilities
% pdf of x
\newcommand{\pdf}{p} % p
\newcommand{\pdfx}{p(\xv)} % p(x)
\newcommand{\pixt}{\pi(\xv~|~ \thetav)} % pi(x|theta), pdf of x given theta
\newcommand{\pixit}[1][i]{\pi\left(\xi[#1] ~|~ \thetav\right)} % pi(x^i|theta), pdf of x given theta
\newcommand{\pixii}[1][i]{\pi\left(\xi[#1]\right)} % pi(x^i), pdf of i-th x

% pdf of (x, y)
\newcommand{\pdfxy}{p(\xv,y)} % p(x, y)
\newcommand{\pdfxyt}{p(\xv, y ~|~ \thetav)} % p(x, y | theta)
\newcommand{\pdfxyit}{p\left(\xi, \yi ~|~ \thetav\right)} % p(x^(i), y^(i) | theta)

% pdf of x given y
\newcommand{\pdfxyk}[1][k]{p(\xv | y= #1)} % p(x | y = k)
\newcommand{\lpdfxyk}[1][k]{\log p(\xv | y= #1)} % log p(x | y = k)
\newcommand{\pdfxiyk}[1][k]{p\left(\xi | y= #1 \right)} % p(x^i | y = k)

% prior probabilities
\newcommand{\pik}[1][k]{\pi_{#1}} % pi_k, prior
\newcommand{\lpik}[1][k]{\log \pi_{#1}} % log pi_k, log of the prior
\newcommand{\pit}{\pi(\thetav)} % Prior probability of parameter theta

% posterior probabilities
\newcommand{\post}{\P(y = 1 ~|~ \xv)} % P(y = 1 | x), post. prob for y=1
\newcommand{\postk}[1][k]{\P(y = #1 ~|~ \xv)} % P(y = k | y), post. prob for y=k
\newcommand{\pidomains}{\pi: \Xspace \rightarrow \unitint} % pi with domain and co-domain
\newcommand{\pibayes}{\pi^{\ast}} % Bayes-optimal classification model
\newcommand{\pixbayes}{\pi^{\ast}(\xv)} % Bayes-optimal classification model
\newcommand{\pix}{\pi(\xv)} % pi(x), P(y = 1 | x)
\newcommand{\piv}{\bm{\pi}} % pi, bold, as vector
\newcommand{\pikx}[1][k]{\pi_{#1}(\xv)} % pi_k(x), P(y = k | x)
\newcommand{\pikxt}[1][k]{\pi_{#1}(\xv ~|~ \thetav)} % pi_k(x | theta), P(y = k | x, theta)
\newcommand{\pixh}{\hat \pi(\xv)} % pi(x) hat, P(y = 1 | x) hat
\newcommand{\pikxh}[1][k]{\hat \pi_{#1}(\xv)} % pi_k(x) hat, P(y = k | x) hat
\newcommand{\pixih}{\hat \pi(\xi)} % pi(x^(i)) with hat
\newcommand{\pikxih}[1][k]{\hat \pi_{#1}(\xi)} % pi_k(x^(i)) with hat
\newcommand{\pdfygxt}{p(y ~|~\xv, \thetav)} % p(y | x, theta)
\newcommand{\pdfyigxit}{p\left(\yi ~|~\xi, \thetav\right)} % p(y^i |x^i, theta)
\newcommand{\lpdfygxt}{\log \pdfygxt } % log p(y | x, theta)
\newcommand{\lpdfyigxit}{\log \pdfyigxit} % log p(y^i |x^i, theta)

% probababilistic
\newcommand{\bayesrulek}[1][k]{\frac{\P(\xv | y= #1) \P(y= #1)}{\P(\xv)}} % Bayes rule
\newcommand{\muk}{\bm{\mu_k}} % mean vector of class-k Gaussian (discr analysis)

% residual and margin
\newcommand{\eps}{\epsilon} % residual, stochastic
\newcommand{\epsv}{\bm{\epsilon}} % residual, stochastic, as vector
\newcommand{\epsi}{\epsilon^{(i)}} % epsilon^i, residual, stochastic
\newcommand{\epsh}{\hat{\epsilon}} % residual, estimated
\newcommand{\epsvh}{\hat{\epsv}} % residual, estimated, vector
\newcommand{\yf}{y \fx} % y f(x), margin
\newcommand{\yfi}{\yi \fxi} % y^i f(x^i), margin
\newcommand{\Sigmah}{\hat \Sigma} % estimated covariance matrix
\newcommand{\Sigmahj}{\hat \Sigma_j} % estimated covariance matrix for the j-th class

% ml - loss, risk, likelihood
\newcommand{\Lyf}{L\left(y, f\right)} % L(y, f), loss function
\newcommand{\Lypi}{L\left(y, \pi\right)} % L(y, pi), loss function
\newcommand{\Lxy}{L\left(y, \fx\right)} % L(y, f(x)), loss function
\newcommand{\Lxyi}{L\left(\yi, \fxi\right)} % loss of observation
\newcommand{\Lxyt}{L\left(y, \fxt\right)} % loss with f parameterized
\newcommand{\Lxyit}{L\left(\yi, \fxit\right)} % loss of observation with f parameterized
\newcommand{\Lxym}{L\left(\yi, f\left(\bm{\tilde{x}}^{(i)} ~|~ \thetav\right)\right)} % loss of observation with f parameterized
\newcommand{\Lpixy}{L\left(y, \pix\right)} % loss in classification
\newcommand{\Lpiv}{L\left(y, \piv\right)} % loss in classification
\newcommand{\Lpixyi}{L\left(\yi, \pixii\right)} % loss of observation in classification
\newcommand{\Lpixyt}{L\left(y, \pixt\right)} % loss with pi parameterized
\newcommand{\Lpixyit}{L\left(\yi, \pixit\right)} % loss of observation with pi parameterized
\newcommand{\Lhxy}{L\left(y, \hx\right)} % L(y, h(x)), loss function on discrete classes
\newcommand{\Lr}{L\left(r\right)} % L(r), loss defined on residual (reg) / margin (classif)
\newcommand{\lone}{|y - \fx|} % L1 loss
\newcommand{\ltwo}{\left(y - \fx\right)^2} % L2 loss
\newcommand{\lbernoullimp}{\ln(1 + \exp(-y \cdot \fx))} % Bernoulli loss for -1, +1 encoding
\newcommand{\lbernoullizo}{- y \cdot \fx + \log(1 + \exp(\fx))} % Bernoulli loss for 0, 1 encoding
\newcommand{\lcrossent}{- y \log \left(\pix\right) - (1 - y) \log \left(1 - \pix\right)} % cross-entropy loss
\newcommand{\lbrier}{\left(\pix - y \right)^2} % Brier score
\newcommand{\risk}{\mathcal{R}} % R, risk
\newcommand{\riskbayes}{\mathcal{R}^\ast}
\newcommand{\riskf}{\risk(f)} % R(f), risk
\newcommand{\riskdef}{\E_{y|\xv}\left(\Lxy \right)} % risk def (expected loss)
\newcommand{\riskt}{\mathcal{R}(\thetav)} % R(theta), risk
\newcommand{\riske}{\mathcal{R}_{\text{emp}}} % R_emp, empirical risk w/o factor 1 / n
\newcommand{\riskeb}{\bar{\mathcal{R}}_{\text{emp}}} % R_emp, empirical risk w/ factor 1 / n
\newcommand{\riskef}{\riske(f)} % R_emp(f)
\newcommand{\risket}{\mathcal{R}_{\text{emp}}(\thetav)} % R_emp(theta)
\newcommand{\riskr}{\mathcal{R}_{\text{reg}}} % R_reg, regularized risk
\newcommand{\riskrt}{\mathcal{R}_{\text{reg}}(\thetav)} % R_reg(theta)
\newcommand{\riskrf}{\riskr(f)} % R_reg(f)
\newcommand{\riskrth}{\hat{\mathcal{R}}_{\text{reg}}(\thetav)} % hat R_reg(theta)
\newcommand{\risketh}{\hat{\mathcal{R}}_{\text{emp}}(\thetav)} % hat R_emp(theta)
\newcommand{\LL}{\mathcal{L}} % L, likelihood
\newcommand{\LLt}{\mathcal{L}(\thetav)} % L(theta), likelihood
\newcommand{\LLtx}{\mathcal{L}(\thetav | \xv)} % L(theta|x), likelihood
\newcommand{\logl}{\ell} % l, log-likelihood
\newcommand{\loglt}{\logl(\thetav)} % l(theta), log-likelihood
\newcommand{\logltx}{\logl(\thetav | \xv)} % l(theta|x), log-likelihood
\newcommand{\errtrain}{\text{err}_{\text{train}}} % training error
\newcommand{\errtest}{\text{err}_{\text{test}}} % test error
\newcommand{\errexp}{\overline{\text{err}_{\text{test}}}} % avg training error

% lm
\newcommand{\thx}{\thetav^\top \xv} % linear model
\newcommand{\olsest}{(\Xmat^\top \Xmat)^{-1} \Xmat^\top \yv} % OLS estimator in LM

% resampling
\newcommand{\ntest}{n_{\mathrm{test}}} % size of the test set
\newcommand{\ntrain}{n_{\mathrm{train}}} % size of the train set
\newcommand{\ntesti}[1][i]{n_{\mathrm{test},#1}} % size of the i-th test set
\newcommand{\ntraini}[1][i]{n_{\mathrm{train},#1}} % size of the i-th train set
\newcommand{\Jtrain}{J_\mathrm{train}} % index vector train data
\newcommand{\Jtest}{J_\mathrm{test}} % index vector test data
\newcommand{\Jtraini}[1][i]{J_{\mathrm{train},#1}} % index vector i-th train dataset
\newcommand{\Jtesti}[1][i]{J_{\mathrm{test},#1}} % index vector i-th test dataset
\newcommand{\Dtraini}[1][i]{\mathcal{D}_{\text{train},#1}} % D_train,i, i-th training set
\newcommand{\Dtesti}[1][i]{\mathcal{D}_{\text{test},#1}} % D_test,i, i-th test set

\newcommand{\JSpace}[1][m]{\nset^{#1}} % space of train indices of size n_train
\newcommand{\JtrainSpace}{\nset^{\ntrain}} % space of train indices of size n_train
\newcommand{\JtestSpace}{\nset^{\ntest}} % space of train indices of size n_test
\newcommand{\yJ}[1][J]{\yv_{#1}} % output vector associated to index J
\newcommand{\yJDef}{\left(y^{(J^{(1)})},\dots,y^{(J^{(m)})}\right)} % def of the output vector associated to index J
\newcommand{\JJ}{\mathcal{J}} % cali-J, set of all splits
\newcommand{\JJset}{\left((\Jtraini[1], \Jtesti[1]),\dots,(\Jtraini[B], \Jtesti[B])\right)} % (Jtrain_1,Jtest_1) ...(Jtrain_B,Jtest_B)
\newcommand{\Itrainlam}{\ind(\Dtrain, \lamv)}
% Generalization error
\newcommand{\GE}{\mathrm{GE}} % GE
\newcommand{\GEh}{\widehat{\GE}} % GE-hat
\newcommand{\GEfull}[1][\ntrain]{\GE(\ind, \lamv, #1, \rho)} % GE full
\newcommand{\GEhholdout}{\GEh_{\Jtrain, \Jtest}(\ind, \lamv, |\Jtrain|, \rho)} % GE hat holdout
\newcommand{\GEhholdouti}[1][i]{\GEh_{\Jtraini[#1], \Jtesti[#1]}(\ind, \lamv, |\Jtraini[#1]|, \rho)} % GE hat holdout i-th set
\newcommand{\GEhlam}{\GEh(\lamv)} % GE-hat(lam)
\newcommand{\GEhlamsubIJrho}{\GEh_{\ind, \JJ, \rho}(\lamv)} % GE-hat_I,J,rho(lam)
\newcommand{\GEhresa}{\GEh(\ind, \JJ, \rho, \lamv)} % GE-hat_I,J,rho(lam)
\newcommand{\GErhoDef}{\lim_{\ntest\rightarrow\infty} \E_{\Dtrain,\Dtest \sim \Pxy} \left[ \rho\left(\yv_{\Jtest}, \FJtestftrain\right)\right]} % GE formal def
\newcommand{\agr}{\mathrm{agr}} % aggregate function
\newcommand{\GEf}{\GE\left(\fh\right)} % GE of a fitted model
\newcommand{\GEfh}{\GEh\left(\fh\right)} % GEh of a fitted model
\newcommand{\GEfL}{\GE\left(\fh, L\right)} % GE of a fitted model wrt loss L
\newcommand{\Lyfhx}{L\left(y, \hat{f}(\xv)\right)} % pointwise loss of fitted model
\newcommand{\GEnf}[1]{GE_n\left(\fh_{#1}\right)} % GE of a fitted model
\newcommand{\GEind}{GE_n\left(\ind_{L, O}\right)} % GE of inducer
\newcommand{\GED}{\GE_{\D}} % GE indexed with data
\newcommand{\EGEn}{EGE_n} % expected GE
\newcommand{\EDn}{\E_{|D| = n}} % expectation wrt data of size n

% performance measure
\newcommand{\rhoL}{\rho_L} % perf. measure derived from pointwise loss
\newcommand{\F}{\bm{F}} % matrix of prediction scores
\newcommand{\Fi}[1][i]{\F^{(#1)}} % i-th row vector of the predscore mat
\newcommand{\FJ}[1][J]{\F_{#1}} % predscore mat idxvec J
\newcommand{\FJf}{\FJ[J,f]} % predscore mat idxvec J and model f
\newcommand{\FJtestfh}{\FJ[\Jtest, \fh]} % predscore mat idxvec Jtest and model f hat
\newcommand{\FJtestftrain}{\F_{\Jtest, \Itrainlam}} % predscore mat idxvec Jtest and model f
\newcommand{\FJtestftraini}[1][i]{\F_{\Jtesti[#1],\ind(\Dtraini[#1], \lamv)}}  % predscore mat i-th idxvec Jtest and model f
\newcommand{\FJfDef}{\left(f(\xv^{(J^{(1)})}),\dots, f(\xv^{(J^{(m)})})\right)} % def of predscore mat idxvec J and model f
\newcommand{\preimageRho}{\bigcup_{m\in\N}\left(\Yspace^m\times\R^{m\times g}\right)} % Set of all datasets times HP space

% ml - ROC
\newcommand{\np}{n_{+}} % no. of positive instances
\newcommand{\nn}{n_{-}} % no. of negative instances
\newcommand{\rn}{\pi_{-}} % proportion negative instances
\newcommand{\rp}{\pi_{+}} % proportion negative instances
  % true/false pos/neg:
\newcommand{\tp}{\# \text{TP}} % true pos
\newcommand{\fap}{\# \text{FP}} % false pos (fp taken for partial derivs)
\newcommand{\tn}{\# \text{TN}} % true neg
\newcommand{\fan}{\# \text{FN}} % false neg

\input{../../latex-math/ml-hpo.tex}

\title{Introduction to Machine Learning}

\begin{document}

\titlemeta{% Chunk title (example: CART, Forests, Boosting, ...), can be empty
Evaluation 
}{% Lecture title  
Measures for Binary Classification: ROC Visualization
}{% Relative path to title page image: Can be empty but must not start with slides/
  figure/eval_mclass_roc_sp_4a
}{% Learning goals, wrapped inside itemize environment
  \item Understand ROC curve
  \item Be able to compute a ROC curve manually
  \item Understand that ROC curve is invariant to class priors at test-time 
  \item Discuss threshold selection
  \item Understand AUC
}

% ------------------------------------------------------------------------------

\begin{vbframe}{Labels: ROC Space}

\begin{itemize}
 \item For comparing classifiers, we characterize them by their TPR and FPR values and plot them in 
 a coordinate system.
 \item We could also use two different ROC metrics which define a trade-off, 
 for instance, TPR and PPV.
\end{itemize}

\lz

\begin{minipage}[c]{0.5\textwidth}
  \begin{knitrout}
    \scriptsize
    \definecolor{shadecolor}{rgb}{0.969, 0.969, 0.969}\color{fgcolor}
    {\centering \includegraphics[width=\textwidth]{figure/eval_mclass_roc_sp_1}}
  \end{knitrout}
\end{minipage}%
\begin{minipage}[c]{0.5\textwidth}
% \includegraphics[width=\textwidth]{figure_man/roc-confmatrix2.png}
\begin{center}
  \small
  \begin{tabular}{cc|cc}
    & & \multicolumn{2}{c}{\bfseries True Class $y$} \\
    & & $+$ & $-$ \\
    \hline
    \bfseries Pred.     & $+$ & TP & FP \\
              $\yh$ & $-$ & FN & TN \\
\end{tabular}
\lz
$$\text{TPR} = \frac{\text{TP}}{\text{TP} + \text{FN}}$$
$$\text{FPR} = \frac{\text{FP}}{\text{FP} + \text{TN}}$$
\end{center}
\end{minipage}

\end{vbframe}

% ------------------------------------------------------------------------------

\begin{vbframe}{Labels: ROC Space}

\begin{itemize}
  \item The best classifier lies on the top-left corner, where FPR equals 0 and 
  TPR is maximal.
  \item The diagonal is worst as it corresponds to a classifier producing random 
  labels (with different proportions). 
\end{itemize}

\lz

\begin{minipage}[c]{0.5\textwidth}
  \begin{itemize}
    \item If each positive $x$ will be randomly classified 
    with 25\% as "pos", $\text{TPR} = 0.25$.
    \item If we assign each negative $x$ randomly to "pos", $\text{FPR} = 0.25$.
  \end{itemize}
\end{minipage}%
\begin{minipage}[c]{0.5\textwidth}
  \centering \includegraphics[width=0.8\textwidth]{figure/eval_mclass_roc_sp_2}
\end{minipage}

\end{vbframe}

% ------------------------------------------------------------------------------

\begin{vbframe}{Labels: ROC Space}

\begin{itemize}
  \item In practice, we should never obtain a classifier below the diagonal.
  \item Inverting the predicted labels ($0 \mapsto 1$ and $1 \mapsto 0$) will 
  result in a reflection at the diagonal. \\
  $\Rightarrow \text{TPR}_{\text{new}} = 1 - \text{TPR}$ and 
  $\text{FPR}_{\text{new}} = 1 - \text{FPR}.$ \\
\end{itemize}

\begin{center}
  \includegraphics[width=0.4\textwidth]{figure/eval_mclass_roc_sp_3}
\end{center}

\end{vbframe}

% ------------------------------------------------------------------------------

\begin{vbframe}{Label Distribution in TPR and FPR}

TPR and FPR (ROC curves) are insensitive to the class distribution in the sense 
that they are not affected by changes in the ratio $\np/\nn$ (at prediction).

\lz

\begin{columns}
  \begin{column}{0.45\textwidth}
  \underline{Example 1}:\\
  Proportion $\np/\nn = 1$\\
  \lz
  {
  \tiny
  \centering
  \tiny
  \begin{tabular}{|l|c|c|}
                  \hline
                 & Actual Positive & Actual Negative \\ \hline
  Pred. Positive & 40            & 25            \\ \hline
  Pred. Negative & 10            & 25           \\ \hline
  \end{tabular}
  }
  
  \medskip
  $\text{MCE} = 35/100 = 0.35$\\
  $\text{TPR} = 0.8$\\ 
  $\text{FPR} = 0.5$ 
\end{column}
\begin{column}{0.45\textwidth} 
  \underline{Example 2}:\\
  Proportion $\np/\nn = 2$\\
  \lz
  {
  \tiny
  \begin{tabular}{|l|c|c|}
                  \hline
                 & Actual Positive & Actual Negative \\ \hline
  Pred. Positive & 80            & 25            \\ \hline
  Pred. Negative & 20            & 25           \\ \hline
  \end{tabular}
  }
  
  \medskip
  $\text{MCE} = 45/150 = 0.3$\\
  $\text{TPR} = 0.8$\\ 
  $\text{FPR} = 0.5$ 
\end{column}
\end{columns}

\lz

Note: If class proportions differ during training, the above is not true. 
Estimated posterior probabilities can change!

\end{vbframe}

% ------------------------------------------------------------------------------

% \begin{vbframe}{Scoring Classifiers}
% \begin{itemize}
% \item A scoring classifier is a model which outputs scores or probabilities, instead of discrete labels, and nearly all modern classifiers can do that.
% \item Thresholding flexibly converts measured probabilities to labels.
%   Predict $1$ (positive class) if $\fxh > c$ else predict $0$.
% \item Normally we could use $c = 0.5$ to convert, but for imbalanced or cost-sensitive situations another threshold could be much better.
% \item After thresholding, any metric defined on labels can be used.
% \end{itemize}
% \begin{center}
% % FIGURE SOURCE: https://docs.google.com/presentation/d/1GmlgtjSCTHgSAveVGf-x1ojAjGP2llPhFKjn_6M4Sig/edit?usp=sharing
% \includegraphics[width=0.5\textwidth]{figure_man/confusion_matrix_measures}
% \end{center}
% \end{vbframe}

% ------------------------------------------------------------------------------

\begin{vbframe}{From Probabilities to Labels: ROC Curve}

Remember: Both probabilistic and scoring classifiers can output classes by 
thresholding:
$$\hx = [\pix \ge c] \quad \text{ or } \quad \hx = [\fx \ge c_f].$$

% \begin{center}
%   \includegraphics{../supervised-classification/figure_man/classifiers.png}
% \end{center}

\textbf{To draw a ROC curve}:

\lz

\begin{minipage}[b]{0.65\textwidth}
  \footnotesize
  \begin{enumerate}
    \item Rank test observations on decreasing score.
    \item Start with $c = 1$, so we start in $(0, 0)$; we predict everything as
    negative.
    \item Iterate through all possible thresholds $c$ and proceed for each
    observation $x$ as follows:
    \begin{itemize}
      \footnotesize
      \item If $x$ is positive, move TPR $1/n_+$ up, \\as we have one TP more.
      \item If $x$ is negative, move FPR $1/n_-$ right, \\as we have one FP 
      more.
    \end{itemize}
  \end{enumerate}
\end{minipage}%
\begin{minipage}[b]{0.35\textwidth}
  \centering
  \includegraphics[width=\textwidth]{figure/eval_mclass_roc_sp_4}
\end{minipage}

\end{vbframe}

% ------------------------------------------------------------------------------

\begin{vbframe}{Drawing ROC Curves}

% new frame for every animation step (rather than framebreak) to prevent plots
% from jumping

\begin{knitrout}\scriptsize
\definecolor{shadecolor}{rgb}{0.969, 0.969, 0.969}\color{fgcolor}

{
% \centering 
\includegraphics[width=0.8\textwidth]{figure/eval_mclass_roc_sp_5} 
}

\end{knitrout}

\vfill

\begin{minipage}[b]{0.3\textwidth}
  $c =$ 0.9\\ 
  $\rightarrow$ TPR = 0.167 \\
  $\rightarrow$ FPR = 0
\end{minipage}%
\begin{minipage}[b]{0.7\textwidth}
  \includegraphics{figure/roc_horizontal_step_1} 
\end{minipage}

\end{vbframe}

% ------------------------------------------------------------------------------

\begin{vbframe}{Drawing ROC Curves}

\begin{knitrout}\scriptsize
\definecolor{shadecolor}{rgb}{0.969, 0.969, 0.969}\color{fgcolor}

{
% \centering 
\includegraphics[width=0.8\textwidth]{figure/eval_mclass_roc_sp_6}
}

\end{knitrout}

\vfill

\begin{minipage}[b]{0.3\textwidth}
  $c =$ 0.85\\ 
  $\rightarrow$ TPR = 0.333 \\
  $\rightarrow$ FPR = 0
\end{minipage}%
\begin{minipage}[b]{0.7\textwidth}
  \includegraphics{figure/roc_horizontal_step_2} 
\end{minipage}

\end{vbframe}

% ------------------------------------------------------------------------------

\begin{vbframe}{Drawing ROC Curves}

\begin{knitrout}\scriptsize
\definecolor{shadecolor}{rgb}{0.969, 0.969, 0.969}\color{fgcolor}

{
% \centering 
\includegraphics[width=0.8\textwidth]{figure/eval_mclass_roc_sp_7}
}

\end{knitrout}

\vfill

\begin{minipage}[b]{0.3\textwidth}
  $c =$ 0.66\\ 
  $\rightarrow$ TPR = 0.5 \\
  $\rightarrow$ FPR = 0
\end{minipage}%
\begin{minipage}[b]{0.7\textwidth}
  \includegraphics{figure/roc_horizontal_step_3} 
\end{minipage}

\end{vbframe}

% ------------------------------------------------------------------------------

\begin{vbframe}{Drawing ROC Curves}

\begin{knitrout}\scriptsize
\definecolor{shadecolor}{rgb}{0.969, 0.969, 0.969}\color{fgcolor}

{
% \centering 
\includegraphics[width=0.8\textwidth]{figure/eval_mclass_roc_sp_8}
}

\end{knitrout}

\vfill

\begin{minipage}[b]{0.3\textwidth}
  $c =$ 0.6\\ 
  $\rightarrow$ TPR = 0.5 \\
  $\rightarrow$ FPR = 0.167
\end{minipage}%
\begin{minipage}[b]{0.7\textwidth}
  \includegraphics{figure/roc_horizontal_step_4} 
\end{minipage}

\end{vbframe}

% ------------------------------------------------------------------------------

\begin{vbframe}{Drawing ROC Curves}

\begin{knitrout}\scriptsize
\definecolor{shadecolor}{rgb}{0.969, 0.969, 0.969}\color{fgcolor}

{
% \centering 
\includegraphics[width=0.8\textwidth]{figure/eval_mclass_roc_sp_9}
}

\end{knitrout}

\vfill

\begin{minipage}[b]{0.3\textwidth}
  $c =$ 0.55\\ 
  $\rightarrow$ TPR = 0.667 \\
  $\rightarrow$ FPR = 0.167
\end{minipage}%
\begin{minipage}[b]{0.7\textwidth}
  \includegraphics{figure/roc_horizontal_step_5} 
\end{minipage}

\end{vbframe}

% ------------------------------------------------------------------------------

\begin{vbframe}{Drawing ROC Curves}

\begin{knitrout}\scriptsize
\definecolor{shadecolor}{rgb}{0.969, 0.969, 0.969}\color{fgcolor}

{
% \centering 
\includegraphics[width=0.8\textwidth]{figure/eval_mclass_roc_sp_10} 
}

\end{knitrout}

\vfill

\begin{minipage}[b]{0.3\textwidth}
  $c =$ 0.3\\ 
  $\rightarrow$ TPR = 0.833 \\
  $\rightarrow$ FPR = 0.5
\end{minipage}%
\begin{minipage}[b]{0.7\textwidth}
  \includegraphics{figure/roc_horizontal_step_6} 
\end{minipage}

\end{vbframe}

% ------------------------------------------------------------------------------

\begin{vbframe}{Drawing ROC Curves}

\begin{knitrout}\scriptsize
\definecolor{shadecolor}{rgb}{0.969, 0.969, 0.969}\color{fgcolor}

{
% \centering 
\includegraphics[width=0.8\textwidth]{figure/eval_mclass_roc_sp_11}
}

\end{knitrout}

\vfill

\begin{minipage}[b]{0.3\textwidth}
	$c =$ 0\\ 
	$\rightarrow$ TPR = 1 \\
	$\rightarrow$ FPR = 1
\end{minipage}%
\begin{minipage}[b]{0.7\textwidth}
	\includegraphics{figure/roc_horizontal_step_7} 
\end{minipage}

\end{vbframe}

% ------------------------------------------------------------------------------

\begin{vbframe}{ROC Curve properties}

\begin{minipage}[c]{0.5\textwidth}
  \begin{itemize}
    \item The closer the curve to the top-left corner, the better.
    \item If ROC curves cross, a different model might be better in different 
    parts of the ROC space.
\end{itemize}
\end{minipage}%
\begin{minipage}[c]{0.5\textwidth}
  \centering 
  \includegraphics[width=\textwidth]{figure/eval_mclass_roc_sp_12}
\end{minipage}

\lz

\begin{itemize}
  \item Small thresholds will very liberally predict the positive class, and 
  result in a potentially higher FPR, but also higher TPR.
  \item High thresholds will very conservatively predict the positive class, 
  and result in a lower FPR and TPR.
  \item As we have not defined the trade-off between false positive and false 
  negative costs, we cannot easily select the "best" threshold. \\
  $\rightarrow$ Visual inspection of all possible results seems useful.
\end{itemize}

% \textcolor{blue}{Include figure?}
% 
% \begin{center}
%   \includegraphics[width=\textwidth]{figure_man/roc-curves2.png}
% \end{center}

% \textcolor{blue}{Include this again or compare it with above?}
%  \begin{itemize}
%    \item Rank test observations on decreasing score
%    \item Set $\alpha=1$, so we start in $(0, 0)$; we predict everything as "neg"
%    \item For each observation $x$ (in the decreasing order).
%    \begin{itemize}
%      \item Reduce threshold, so prediction for next observation changes
%      \item If $x$ is "pos", move TPR $1/n_+$ up, as we have one TP more
%      \item If $x$ is "neg", move FPR $1/n_-$ right, as we have one FP more
%    \end{itemize}
%  \end{itemize}

\end{vbframe}

\begin{vbframe}{Choosing Threshold / Operating point}

Often done visually and post-hoc, as class imbalances or costs are unknown a-priori.

\lz

\begin{columns}
\begin{column}{0.6\textwidth}
\begin{itemize}
  \item Identify non-dominated points 
  \item Assess TPR / FPR 
  \item Decide which combo is best for task
  \item Pick associated threshold
\end{itemize}
\end{column}

\begin{column}{0.4\textwidth}
\begin{center}
  \includegraphics[width=\textwidth,trim={1.5cm 0 0 1.5cm},clip]{figure/eval_mclass_roc_sp_4a.pdf}
\end{center}
\end{column}
\end{columns}
% \begin{itemize}
  % \item Optimal threshold is mean of thresholds leading to that point.
% \end{itemize}
\end{vbframe}

% ------------------------------------------------------------------------------

% \begin{vbframe}{ROC Curve}
% \begin{itemize}
%   \item The closer the curve to the top-left corner, the better
%   \item If ROC curves cross, a different model can be better in different parts 
%   of the ROC space
% \end{itemize}
% \begin{knitrout}\scriptsize
% \definecolor{shadecolor}{rgb}{0.969, 0.969, 0.969}\color{fgcolor}
% 
% {\centering \includegraphics[width=.65\textwidth]{figure/eval_mclass_roc_sp_12} 
% 
% }
% 
% \end{knitrout}
% \end{vbframe}

% ------------------------------------------------------------------------------

\begin{vbframe}{AUC: Area Under ROC Curve}

\begin{itemize}
  % \small
  \item AUC $\in [0,1]$ is a single metric to evaluate scoring classifiers --
  independent of the chosen threshold.
  \begin{itemize}
    \item AUC = 1: perfect classifier
    \item AUC = 0.5: random, non-discriminant classifier
    \item AUC = 0: perfect, with inverted labels
  \end{itemize}
  % \item The AUC is directly related to the Gini coefficient $G$: 
  % $\text{AUC} = 0.5 \cdot (G + 1)$. 
  % \textcolor{blue}{@BB: check if correct/helpful here}
\end{itemize}

\begin{center}
  \includegraphics[width=0.5\textwidth]{figure/eval_mclass_roc_sp_12_1}
\end{center}

\end{vbframe}

% ------------------------------------------------------------------------------

\begin{vbframe}{AUC as a rank-based metric}

\begin{itemize}
  \small
  \item We can also interpret the AUC as the probability of our classifier 
  ranking a random positive observation higher than a random negative one.
  \item A perfect classifier will rank all positive above all negative 
  observations, achieving AUC = 1.
\end{itemize}

\begin{center}
% FIGURE SOURCE: https://docs.google.com/drawings/d/1flfi73s8qr53-ZE6oq4qRGIG-sccpBp2cSfw1Stxh8I/edit
  \includegraphics[width=0.55\textwidth]{figure_man/auc_interpretation_new}
  \includegraphics[width=0.4\textwidth] {figure/fig-eval_mwu_ranking}
\end{center}

\end{vbframe}

% 
% \begin{vbframe}{AUC: Area Under ROC Curve}
% Interpretation: Probability that classifier ranks a random positive higher than a random negative observation
% 
% \begin{center}
% % FIGURE SOURCE: https://docs.google.com/presentation/d/1xj9_84181bqFpr0EMqdGHE6dUf_vAf1qcs9z-siUsCw/edit?usp=sharing
% \includegraphics[width=0.8\textwidth,page=1]{figure_man/auc_interpretation.pdf}
% \end{center}
% 
% \end{vbframe}


% \begin{vbframe}{Partial AUC}
% \begin{itemize}
%   \item Sometimes it can be useful to look at a \href{http://journals.sagepub.com/doi/pdf/10.1177/0272989X8900900307}{specific region under the ROC curve}  $\Rightarrow$ partial AUC (pAUC).
%   \item Examples: focus on a region with low FPR or a region with high TPR:
% \end{itemize}
% 
% \begin{knitrout}\scriptsize
% \definecolor{shadecolor}{rgb}{0.969, 0.969, 0.969}\color{fgcolor}
% 
% {\centering \includegraphics[width=0.9\textwidth]{figure/eval_mclass_roc_sp_13} 
% 
% }

% \end{knitrout}

% ------------------------------------------------------------------------------

\endlecture
\end{document}
