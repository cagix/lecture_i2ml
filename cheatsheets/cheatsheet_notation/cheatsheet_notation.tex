\documentclass{beamer}


\usepackage[orientation=landscape,size=a0,scale=1.4,debug]{beamerposter}
\mode<presentation>{\usetheme{mlr}}


\usepackage[utf8]{inputenc} % UTF-8
\usepackage[english]{babel} % Language
\usepackage{hyperref} % Hyperlinks
\usepackage{ragged2e} % Text position
\usepackage[export]{adjustbox} % Image position
\usepackage[most]{tcolorbox}
\usepackage{amsmath}
\usepackage{mathtools}
\usepackage{dsfont}
\usepackage{verbatim}
\usepackage{amsmath}
\usepackage{amsfonts}
\usepackage{csquotes}
\usepackage{multirow}
\usepackage{longtable}
\usepackage{enumerate}
\usepackage[absolute,overlay]{textpos}
\usepackage{psfrag}
\usepackage{algorithm}
\usepackage{algpseudocode}
\usepackage{eqnarray}
\usepackage{arydshln}
\usepackage{tabularx}
\usepackage{placeins}
\usepackage{tikz}
\usepackage{setspace}
\usepackage{colortbl}
\usepackage{mathtools}
\usepackage{wrapfig}
\usepackage{bm}

% dependencies: amsmath, amssymb, dsfont
% math spaces
\ifdefined\N
\renewcommand{\N}{\mathds{N}} % N, naturals
\else \newcommand{\N}{\mathds{N}} \fi
\newcommand{\Z}{\mathds{Z}} % Z, integers
\newcommand{\Q}{\mathds{Q}} % Q, rationals
\newcommand{\R}{\mathds{R}} % R, reals
\ifdefined\C
\renewcommand{\C}{\mathds{C}} % C, complex
\else \newcommand{\C}{\mathds{C}} \fi
\newcommand{\continuous}{\mathcal{C}} % C, space of continuous functions
\newcommand{\M}{\mathcal{M}} % machine numbers
\newcommand{\epsm}{\epsilon_m} % maximum error

% counting / finite sets
\newcommand{\setzo}{\{0, 1\}} % set 0, 1
\newcommand{\setmp}{\{-1, +1\}} % set -1, 1
\newcommand{\unitint}{[0, 1]} % unit interval

% basic math stuff
\newcommand{\xt}{\tilde x} % x tilde
\DeclareMathOperator*{\argmax}{arg\,max} % argmax
\DeclareMathOperator*{\argmin}{arg\,min} % argmin
\newcommand{\argminlim}{\mathop{\mathrm{arg\,min}}\limits} % argmax with limits
\newcommand{\argmaxlim}{\mathop{\mathrm{arg\,max}}\limits} % argmin with limits
\newcommand{\sign}{\operatorname{sign}} % sign, signum
\newcommand{\I}{\mathbb{I}} % I, indicator
\newcommand{\order}{\mathcal{O}} % O, order
\newcommand{\bigO}{\mathcal{O}} % Big-O Landau
\newcommand{\littleo}{{o}} % Little-o Landau
\newcommand{\pd}[2]{\frac{\partial{#1}}{\partial #2}} % partial derivative
\newcommand{\floorlr}[1]{\left\lfloor #1 \right\rfloor} % floor
\newcommand{\ceillr}[1]{\left\lceil #1 \right\rceil} % ceiling
\newcommand{\indep}{\perp \!\!\! \perp} % independence symbol

% sums and products
\newcommand{\sumin}{\sum\limits_{i=1}^n} % summation from i=1 to n
\newcommand{\sumim}{\sum\limits_{i=1}^m} % summation from i=1 to m
\newcommand{\sumjn}{\sum\limits_{j=1}^n} % summation from j=1 to p
\newcommand{\sumjp}{\sum\limits_{j=1}^p} % summation from j=1 to p
\newcommand{\sumik}{\sum\limits_{i=1}^k} % summation from i=1 to k
\newcommand{\sumkg}{\sum\limits_{k=1}^g} % summation from k=1 to g
\newcommand{\sumjg}{\sum\limits_{j=1}^g} % summation from j=1 to g
\newcommand{\summM}{\sum\limits_{m=1}^M} % summation from m=1 to M
\newcommand{\meanin}{\frac{1}{n} \sum\limits_{i=1}^n} % mean from i=1 to n
\newcommand{\meanim}{\frac{1}{m} \sum\limits_{i=1}^m} % mean from i=1 to n
\newcommand{\meankg}{\frac{1}{g} \sum\limits_{k=1}^g} % mean from k=1 to g
\newcommand{\meanmM}{\frac{1}{M} \sum\limits_{m=1}^M} % mean from m=1 to M
\newcommand{\prodin}{\prod\limits_{i=1}^n} % product from i=1 to n
\newcommand{\prodkg}{\prod\limits_{k=1}^g} % product from k=1 to g
\newcommand{\prodjp}{\prod\limits_{j=1}^p} % product from j=1 to p

% linear algebra
\newcommand{\one}{\bm{1}} % 1, unitvector
\newcommand{\zero}{\mathbf{0}} % 0-vector
\newcommand{\id}{\bm{I}} % I, identity
\newcommand{\diag}{\operatorname{diag}} % diag, diagonal
\newcommand{\trace}{\operatorname{tr}} % tr, trace
\newcommand{\spn}{\operatorname{span}} % span
\newcommand{\scp}[2]{\left\langle #1, #2 \right\rangle} % <.,.>, scalarproduct
\newcommand{\mat}[1]{\begin{pmatrix} #1 \end{pmatrix}} % short pmatrix command
\newcommand{\Amat}{\mathbf{A}} % matrix A
\newcommand{\Deltab}{\mathbf{\Delta}} % error term for vectors

% basic probability + stats
\renewcommand{\P}{\mathds{P}} % P, probability
\newcommand{\E}{\mathds{E}} % E, expectation
\newcommand{\var}{\mathsf{Var}} % Var, variance
\newcommand{\cov}{\mathsf{Cov}} % Cov, covariance
\newcommand{\corr}{\mathsf{Corr}} % Corr, correlation
\newcommand{\normal}{\mathcal{N}} % N of the normal distribution
\newcommand{\iid}{\overset{i.i.d}{\sim}} % dist with i.i.d superscript
\newcommand{\distas}[1]{\overset{#1}{\sim}} % ... is distributed as ...

% machine learning
\newcommand{\Xspace}{\mathcal{X}} % X, input space
\newcommand{\Yspace}{\mathcal{Y}} % Y, output space
\newcommand{\Zspace}{\mathcal{Z}} % Z, space of sampled datapoints
\newcommand{\nset}{\{1, \ldots, n\}} % set from 1 to n
\newcommand{\pset}{\{1, \ldots, p\}} % set from 1 to p
\newcommand{\gset}{\{1, \ldots, g\}} % set from 1 to g
\newcommand{\Pxy}{\mathbb{P}_{xy}} % P_xy
\newcommand{\Exy}{\mathbb{E}_{xy}} % E_xy: Expectation over random variables xy
\newcommand{\xv}{\mathbf{x}} % vector x (bold)
\newcommand{\xtil}{\tilde{\mathbf{x}}} % vector x-tilde (bold)
\newcommand{\yv}{\mathbf{y}} % vector y (bold)
\newcommand{\xy}{(\xv, y)} % observation (x, y)
\newcommand{\xvec}{\left(x_1, \ldots, x_p\right)^\top} % (x1, ..., xp)
\newcommand{\Xmat}{\mathbf{X}} % Design matrix
\newcommand{\allDatasets}{\mathds{D}} % The set of all datasets
\newcommand{\allDatasetsn}{\mathds{D}_n}  % The set of all datasets of size n
\newcommand{\D}{\mathcal{D}} % D, data
\newcommand{\Dn}{\D_n} % D_n, data of size n
\newcommand{\Dtrain}{\mathcal{D}_{\text{train}}} % D_train, training set
\newcommand{\Dtest}{\mathcal{D}_{\text{test}}} % D_test, test set
\newcommand{\xyi}[1][i]{\left(\xv^{(#1)}, y^{(#1)}\right)} % (x^i, y^i), i-th observation
\newcommand{\Dset}{\left( \xyi[1], \ldots, \xyi[n]\right)} % {(x1,y1)), ..., (xn,yn)}, data
\newcommand{\defAllDatasetsn}{(\Xspace \times \Yspace)^n} % Def. of the set of all datasets of size n
\newcommand{\defAllDatasets}{\bigcup_{n \in \N}(\Xspace \times \Yspace)^n} % Def. of the set of all datasets
\newcommand{\xdat}{\left\{ \xv^{(1)}, \ldots, \xv^{(n)}\right\}} % {x1, ..., xn}, input data
\newcommand{\ydat}{\left\{ \yv^{(1)}, \ldots, \yv^{(n)}\right\}} % {y1, ..., yn}, input data
\newcommand{\yvec}{\left(y^{(1)}, \hdots, y^{(n)}\right)^\top} % (y1, ..., yn), vector of outcomes
\newcommand{\greekxi}{\xi} % Greek letter xi
\renewcommand{\xi}[1][i]{\xv^{(#1)}} % x^i, i-th observed value of x
\newcommand{\yi}[1][i]{y^{(#1)}} % y^i, i-th observed value of y
\newcommand{\xivec}{\left(x^{(i)}_1, \ldots, x^{(i)}_p\right)^\top} % (x1^i, ..., xp^i), i-th observation vector
\newcommand{\xj}{\xv_j} % x_j, j-th feature
\newcommand{\xjvec}{\left(x^{(1)}_j, \ldots, x^{(n)}_j\right)^\top} % (x^1_j, ..., x^n_j), j-th feature vector
\newcommand{\phiv}{\mathbf{\phi}} % Basis transformation function phi
\newcommand{\phixi}{\mathbf{\phi}^{(i)}} % Basis transformation of xi: phi^i := phi(xi)

%%%%%% ml - models general
\newcommand{\lamv}{\bm{\lambda}} % lambda vector, hyperconfiguration vector
\newcommand{\Lam}{\bm{\Lambda}}	 % Lambda, space of all hpos
% Inducer / Inducing algorithm
\newcommand{\preimageInducer}{\left(\defAllDatasets\right)\times\Lam} % Set of all datasets times the hyperparameter space
\newcommand{\preimageInducerShort}{\allDatasets\times\Lam} % Set of all datasets times the hyperparameter space
% Inducer / Inducing algorithm
\newcommand{\ind}{\mathcal{I}} % Inducer, inducing algorithm, learning algorithm

% continuous prediction function f
\newcommand{\ftrue}{f_{\text{true}}}  % True underlying function (if a statistical model is assumed)
\newcommand{\ftruex}{\ftrue(\xv)} % True underlying function (if a statistical model is assumed)
\newcommand{\fx}{f(\xv)} % f(x), continuous prediction function
\newcommand{\fdomains}{f: \Xspace \rightarrow \R^g} % f with domain and co-domain
\newcommand{\Hspace}{\mathcal{H}} % hypothesis space where f is from
\newcommand{\fbayes}{f^{\ast}} % Bayes-optimal model
\newcommand{\fxbayes}{f^{\ast}(\xv)} % Bayes-optimal model
\newcommand{\fkx}[1][k]{f_{#1}(\xv)} % f_j(x), discriminant component function
\newcommand{\fh}{\hat{f}} % f hat, estimated prediction function
\newcommand{\fxh}{\fh(\xv)} % fhat(x)
\newcommand{\fxt}{f(\xv ~|~ \thetav)} % f(x | theta)
\newcommand{\fxi}{f\left(\xv^{(i)}\right)} % f(x^(i))
\newcommand{\fxih}{\hat{f}\left(\xv^{(i)}\right)} % f(x^(i))
\newcommand{\fxit}{f\left(\xv^{(i)} ~|~ \thetav\right)} % f(x^(i) | theta)
\newcommand{\fhD}{\fh_{\D}} % fhat_D, estimate of f based on D
\newcommand{\fhDtrain}{\fh_{\Dtrain}} % fhat_Dtrain, estimate of f based on D
\newcommand{\fhDnlam}{\fh_{\Dn, \lamv}} %model learned on Dn with hp lambda
\newcommand{\fhDlam}{\fh_{\D, \lamv}} %model learned on D with hp lambda
\newcommand{\fhDnlams}{\fh_{\Dn, \lamv^\ast}} %model learned on Dn with optimal hp lambda
\newcommand{\fhDlams}{\fh_{\D, \lamv^\ast}} %model learned on D with optimal hp lambda

% discrete prediction function h
\newcommand{\hx}{h(\xv)} % h(x), discrete prediction function
\newcommand{\hh}{\hat{h}} % h hat
\newcommand{\hxh}{\hat{h}(\xv)} % hhat(x)
\newcommand{\hxt}{h(\xv | \thetav)} % h(x | theta)
\newcommand{\hxi}{h\left(\xi\right)} % h(x^(i))
\newcommand{\hxit}{h\left(\xi ~|~ \thetav\right)} % h(x^(i) | theta)
\newcommand{\hbayes}{h^{\ast}} % Bayes-optimal classification model
\newcommand{\hxbayes}{h^{\ast}(\xv)} % Bayes-optimal classification model

% yhat
\newcommand{\yh}{\hat{y}} % yhat for prediction of target
\newcommand{\yih}{\hat{y}^{(i)}} % yhat^(i) for prediction of ith targiet
\newcommand{\resi}{\yi- \yih}

% theta
\newcommand{\thetah}{\hat{\theta}} % theta hat
\newcommand{\thetav}{\bm{\theta}} % theta vector
\newcommand{\thetavh}{\bm{\hat\theta}} % theta vector hat
\newcommand{\thetat}[1][t]{\thetav^{[#1]}} % theta^[t] in optimization
\newcommand{\thetatn}[1][t]{\thetav^{[#1 +1]}} % theta^[t+1] in optimization
\newcommand{\thetahDnlam}{\thetavh_{\Dn, \lamv}} %theta learned on Dn with hp lambda
\newcommand{\thetahDlam}{\thetavh_{\D, \lamv}} %theta learned on D with hp lambda
\newcommand{\mint}{\min_{\thetav \in \Theta}} % min problem theta
\newcommand{\argmint}{\argmin_{\thetav \in \Theta}} % argmin theta

% densities + probabilities
% pdf of x
\newcommand{\pdf}{p} % p
\newcommand{\pdfx}{p(\xv)} % p(x)
\newcommand{\pixt}{\pi(\xv~|~ \thetav)} % pi(x|theta), pdf of x given theta
\newcommand{\pixit}[1][i]{\pi\left(\xi[#1] ~|~ \thetav\right)} % pi(x^i|theta), pdf of x given theta
\newcommand{\pixii}[1][i]{\pi\left(\xi[#1]\right)} % pi(x^i), pdf of i-th x

% pdf of (x, y)
\newcommand{\pdfxy}{p(\xv,y)} % p(x, y)
\newcommand{\pdfxyt}{p(\xv, y ~|~ \thetav)} % p(x, y | theta)
\newcommand{\pdfxyit}{p\left(\xi, \yi ~|~ \thetav\right)} % p(x^(i), y^(i) | theta)

% pdf of x given y
\newcommand{\pdfxyk}[1][k]{p(\xv | y= #1)} % p(x | y = k)
\newcommand{\lpdfxyk}[1][k]{\log p(\xv | y= #1)} % log p(x | y = k)
\newcommand{\pdfxiyk}[1][k]{p\left(\xi | y= #1 \right)} % p(x^i | y = k)

% prior probabilities
\newcommand{\pik}[1][k]{\pi_{#1}} % pi_k, prior
\newcommand{\lpik}[1][k]{\log \pi_{#1}} % log pi_k, log of the prior
\newcommand{\pit}{\pi(\thetav)} % Prior probability of parameter theta

% posterior probabilities
\newcommand{\post}{\P(y = 1 ~|~ \xv)} % P(y = 1 | x), post. prob for y=1
\newcommand{\postk}[1][k]{\P(y = #1 ~|~ \xv)} % P(y = k | y), post. prob for y=k
\newcommand{\pidomains}{\pi: \Xspace \rightarrow \unitint} % pi with domain and co-domain
\newcommand{\pibayes}{\pi^{\ast}} % Bayes-optimal classification model
\newcommand{\pixbayes}{\pi^{\ast}(\xv)} % Bayes-optimal classification model
\newcommand{\pix}{\pi(\xv)} % pi(x), P(y = 1 | x)
\newcommand{\piv}{\bm{\pi}} % pi, bold, as vector
\newcommand{\pikx}[1][k]{\pi_{#1}(\xv)} % pi_k(x), P(y = k | x)
\newcommand{\pikxt}[1][k]{\pi_{#1}(\xv ~|~ \thetav)} % pi_k(x | theta), P(y = k | x, theta)
\newcommand{\pixh}{\hat \pi(\xv)} % pi(x) hat, P(y = 1 | x) hat
\newcommand{\pikxh}[1][k]{\hat \pi_{#1}(\xv)} % pi_k(x) hat, P(y = k | x) hat
\newcommand{\pixih}{\hat \pi(\xi)} % pi(x^(i)) with hat
\newcommand{\pikxih}[1][k]{\hat \pi_{#1}(\xi)} % pi_k(x^(i)) with hat
\newcommand{\pdfygxt}{p(y ~|~\xv, \thetav)} % p(y | x, theta)
\newcommand{\pdfyigxit}{p\left(\yi ~|~\xi, \thetav\right)} % p(y^i |x^i, theta)
\newcommand{\lpdfygxt}{\log \pdfygxt } % log p(y | x, theta)
\newcommand{\lpdfyigxit}{\log \pdfyigxit} % log p(y^i |x^i, theta)

% probababilistic
\newcommand{\bayesrulek}[1][k]{\frac{\P(\xv | y= #1) \P(y= #1)}{\P(\xv)}} % Bayes rule
\newcommand{\muk}{\bm{\mu_k}} % mean vector of class-k Gaussian (discr analysis)

% residual and margin
\newcommand{\eps}{\epsilon} % residual, stochastic
\newcommand{\epsv}{\bm{\epsilon}} % residual, stochastic, as vector
\newcommand{\epsi}{\epsilon^{(i)}} % epsilon^i, residual, stochastic
\newcommand{\epsh}{\hat{\epsilon}} % residual, estimated
\newcommand{\epsvh}{\hat{\epsv}} % residual, estimated, vector
\newcommand{\yf}{y \fx} % y f(x), margin
\newcommand{\yfi}{\yi \fxi} % y^i f(x^i), margin
\newcommand{\Sigmah}{\hat \Sigma} % estimated covariance matrix
\newcommand{\Sigmahj}{\hat \Sigma_j} % estimated covariance matrix for the j-th class

% ml - loss, risk, likelihood
\newcommand{\Lyf}{L\left(y, f\right)} % L(y, f), loss function
\newcommand{\Lypi}{L\left(y, \pi\right)} % L(y, pi), loss function
\newcommand{\Lxy}{L\left(y, \fx\right)} % L(y, f(x)), loss function
\newcommand{\Lxyi}{L\left(\yi, \fxi\right)} % loss of observation
\newcommand{\Lxyt}{L\left(y, \fxt\right)} % loss with f parameterized
\newcommand{\Lxyit}{L\left(\yi, \fxit\right)} % loss of observation with f parameterized
\newcommand{\Lxym}{L\left(\yi, f\left(\bm{\tilde{x}}^{(i)} ~|~ \thetav\right)\right)} % loss of observation with f parameterized
\newcommand{\Lpixy}{L\left(y, \pix\right)} % loss in classification
\newcommand{\Lpiv}{L\left(y, \piv\right)} % loss in classification
\newcommand{\Lpixyi}{L\left(\yi, \pixii\right)} % loss of observation in classification
\newcommand{\Lpixyt}{L\left(y, \pixt\right)} % loss with pi parameterized
\newcommand{\Lpixyit}{L\left(\yi, \pixit\right)} % loss of observation with pi parameterized
\newcommand{\Lhxy}{L\left(y, \hx\right)} % L(y, h(x)), loss function on discrete classes
\newcommand{\Lr}{L\left(r\right)} % L(r), loss defined on residual (reg) / margin (classif)
\newcommand{\lone}{|y - \fx|} % L1 loss
\newcommand{\ltwo}{\left(y - \fx\right)^2} % L2 loss
\newcommand{\lbernoullimp}{\ln(1 + \exp(-y \cdot \fx))} % Bernoulli loss for -1, +1 encoding
\newcommand{\lbernoullizo}{- y \cdot \fx + \log(1 + \exp(\fx))} % Bernoulli loss for 0, 1 encoding
\newcommand{\lcrossent}{- y \log \left(\pix\right) - (1 - y) \log \left(1 - \pix\right)} % cross-entropy loss
\newcommand{\lbrier}{\left(\pix - y \right)^2} % Brier score
\newcommand{\risk}{\mathcal{R}} % R, risk
\newcommand{\riskbayes}{\mathcal{R}^\ast}
\newcommand{\riskf}{\risk(f)} % R(f), risk
\newcommand{\riskdef}{\E_{y|\xv}\left(\Lxy \right)} % risk def (expected loss)
\newcommand{\riskt}{\mathcal{R}(\thetav)} % R(theta), risk
\newcommand{\riske}{\mathcal{R}_{\text{emp}}} % R_emp, empirical risk w/o factor 1 / n
\newcommand{\riskeb}{\bar{\mathcal{R}}_{\text{emp}}} % R_emp, empirical risk w/ factor 1 / n
\newcommand{\riskef}{\riske(f)} % R_emp(f)
\newcommand{\risket}{\mathcal{R}_{\text{emp}}(\thetav)} % R_emp(theta)
\newcommand{\riskr}{\mathcal{R}_{\text{reg}}} % R_reg, regularized risk
\newcommand{\riskrt}{\mathcal{R}_{\text{reg}}(\thetav)} % R_reg(theta)
\newcommand{\riskrf}{\riskr(f)} % R_reg(f)
\newcommand{\riskrth}{\hat{\mathcal{R}}_{\text{reg}}(\thetav)} % hat R_reg(theta)
\newcommand{\risketh}{\hat{\mathcal{R}}_{\text{emp}}(\thetav)} % hat R_emp(theta)
\newcommand{\LL}{\mathcal{L}} % L, likelihood
\newcommand{\LLt}{\mathcal{L}(\thetav)} % L(theta), likelihood
\newcommand{\LLtx}{\mathcal{L}(\thetav | \xv)} % L(theta|x), likelihood
\newcommand{\logl}{\ell} % l, log-likelihood
\newcommand{\loglt}{\logl(\thetav)} % l(theta), log-likelihood
\newcommand{\logltx}{\logl(\thetav | \xv)} % l(theta|x), log-likelihood
\newcommand{\errtrain}{\text{err}_{\text{train}}} % training error
\newcommand{\errtest}{\text{err}_{\text{test}}} % test error
\newcommand{\errexp}{\overline{\text{err}_{\text{test}}}} % avg training error

% lm
\newcommand{\thx}{\thetav^\top \xv} % linear model
\newcommand{\olsest}{(\Xmat^\top \Xmat)^{-1} \Xmat^\top \yv} % OLS estimator in LM



\title{I2ML :\,: BASICS} % Package title in header, \, adds thin space between ::
\newcommand{\packagedescription}{ \invisible{x} % Package description in header
	% The \textbf{I2ML}: Introduction to Machine Learning course offers an introductory and applied overview of "supervised" Machine Learning. It is organized as a digital lecture.
}

\newlength{\columnheight} % Adjust depending on header height
\setlength{\columnheight}{84cm} 

\newtcolorbox{codebox}{%
	sharp corners,
	leftrule=0pt,
	rightrule=0pt,
	toprule=0pt,
	bottomrule=0pt,
	hbox}

\newtcolorbox{codeboxmultiline}[1][]{%
	sharp corners,
	leftrule=0pt,
	rightrule=0pt,
	toprule=0pt,
	bottomrule=0pt,
	#1}
	

	
\begin{document}
\begin{frame}[fragile]{}
\vspace{-8ex}
\begin{columns}
	\begin{column}{.31\textwidth}
		\begin{beamercolorbox}[center]{postercolumn}
			\begin{minipage}{.98\textwidth}
				\parbox[t][\columnheight]{\textwidth}{
%%%%%%%%%%%%%%%%%%%%%%%%%%%%%%%%%%%%%%%%%%%%%%%%%%%%%%%%%%%%%%%%%%%%%%%%%%%%%%%%
% First Column begin
%-------------------------------------------------------------------------------
% Data
%-------------------------------------------------------------------------------
\begin{myblock}{Data}
 $\Xspace \subseteq \R^p$ : $p$-dimensional \textbf{feature space} / input space\\ 
Usually we assume categorical features to be numerically encoded.\\
%sually we assume $\Xspace \equiv \R^p$, but sometimes, dimensions may be \\  
%bounded (e.g., for categorical or non-negative features.)    \\

$\Yspace$ : \textbf{target space} \\ 
e.g.: $\Yspace = \R$ for regression, $\Yspace = \setzo$ or $\Yspace = \setmp$ for binary classification, $\Yspace = \gset$ for multi-class classification with $g$ classes\\

$\xv = \xvec \in \Xspace$ : \textbf{feature vector} / covariate vector\\ 
 
$y \in \Yspace$ : \textbf{target variable} / output variable \\
Concrete samples are called labels \\

$\xyi \in \Xspace\times \Yspace$ : $i$ -th \textbf{observation} / sample / instance / example\\

$\allDatasets = \defAllDatasets$ : \textbf{set of all finite data sets} \\

$\allDatasetsn = \defAllDatasetsn \subseteq \allDatasets$ : \textbf{set of all finite data sets of size $n$} \\

$\D = \Dset \in \allDatasetsn $ : \textbf{data set} of size $n$.
An n-tuple, a family indexed by $\{1, \dots, n\}$. 
We use $\D_n$ to emphasize its size.\\
 
$\Dtrain$, $\Dtest \subseteq \D$ : \textbf{data sets for training and testing} \\ 
Often: $\D = \Dtrain ~ \dot{\cup} ~ \Dtest$\\
 

$\P_{xy}$ : \textbf{joint probability distribution on} $\Xspace \times \Yspace$ \\


\underline{Classification}\\

$o_k(y) = \I(y = k) \in \{0,1\}$: multiclass one-hot encoding, if $y$ is class k\\ 
We write $\bold{o}(y)$ for the g-length encoding vector and $o_k^{(i)} =  o_k(\yi)$\\

$\pi_k = \P(y = k)$:\textbf{ prior probability} for class $k$ \\
In case of binary labels we might abbreviate: $\pi = \P(y = 1)$.
  
\end{myblock}
%-------------------------------------------------------------------------------
% Model and Learner
%-------------------------------------------------------------------------------
\begin{myblock}{Model and Learner}
    \textbf{Model} / Hypothesis: $f : \Xspace \rightarrow \R^g$ maps features to predictions, often parametrized by $\thetav \in \Theta$ (then we write $f_{\thetav}(\xv)$ or $f(\xv | \thetav)$). \\

% $\fx$ or $\fxt \in \R$ or $\R^g$ : prediction function (\textbf{model}) \\ %learned from data
% \hspace*{1ex}We might suppress $\thetav$ in notation. \\

$\Theta \subseteq \R^d$ : \textbf{parameter space} \\
  
$\thetav = (\theta_1, \theta_2, ..., \theta_d) \in \Theta$: model \textbf{parameter} vector\\
Some models may traditionally use different symbols. \\

$\Hspace = \{f : \Xspace \rightarrow \R^g ~|~ f \text{ belongs to a certain functional family}\}$ : \\ \textbf{Hypothesis space} -- set of functions to which we restrict learning
				
\end{myblock}\vfill
% End First Column
%%%%%%%%%%%%%%%%%%%%%%%%%%%%%%%%%%%%%%%%%%%%%%%%%%%%%%%%%%%%%%%%%%%%%%%%%%%%%%%%
				}
			\end{minipage}
		\end{beamercolorbox}
	\end{column}
	\begin{column}{.31\textwidth}
		\begin{beamercolorbox}[center]{postercolumn}
			\begin{minipage}{.98\textwidth}
				\parbox[t][\columnheight]{\textwidth}{
%%%%%%%%%%%%%%%%%%%%%%%%%%%%%%%%%%%%%%%%%%%%%%%%%%%%%%%%%%%%%%%%%%%%%%%%%%%%%%%%
% Begin Second Column
\begin{myblock}{} \vspace{-4ex}

\textbf{Learner} / Inducer $\inducer: \preimageInducerShort \rightarrow \Hspace$  takes a training set  $\Dtrain \in \allDatasets$, produces model $f : \Xspace \rightarrow \R^g$, with hyperparam. configuration $\bm{\lambda} \in \bm{\Lambda}$.\\
We also write $\inducer: \preimageInducerShort \rightarrow \Theta$ or $\ind_{\bm{\lambda} }: \allDatasets \rightarrow \Theta$ \\

$\bm{\Lambda} = \bm{\Lambda_1} \times \bm{\Lambda_2} \times ... \times \bm{\Lambda_}{\ell} \subseteq \R^{\ell}$: %, where $\bm{\Lambda_j} = (a_j, b_j), \quad a_j, b_j \in \R,$ \\$j = 1, 2, ... , \ll$ : 
\textbf{hyperparameter space} \\
$\bm{\Lambda_j} $ are usually bounded real or integer intervals or a finite categorical set\\
% $\R$, intervals in $\R$ or intervals in $\N$\\

$\bm{\lambda} = (\lambda_1, \lambda_2, ..., \lambda_{\ell}) \in \bm{\Lambda}$ : \textbf{hyperparameter configuration} \\



% $\LLt$ and $\llt = \log\LLt$ : \textbf{likelihood} and \textbf{log-likelihood} for \\ parameter $\thetav$ \\
% These are based on a statistical model.\\

$r = y - \fx$ or $r^{(i)} = \yi - \fxi$ : ($i$-th) \textbf{residual} in regression\\

%????????????????????????????????


\underline{Classification}\\

$\pikx : \Xspace \rightarrow [0,1]$ \textbf{probability prediction} for class $k$, approximates $\postk$;
% $\postk \in [0, 1]$: \textbf{posterior probability} for class $k$, \\ given $\xv$ (
for binary we abbreviate with $\pix$ for $\post$.\\
 
$\fkx: \Xspace \rightarrow \R$: \textbf{scoring} / discriminant \textbf{function} for class $k$;\\
for binary we use $\fx = f_{1}(\xv) - f_{2}(\xv)$\\
 
$\hx: \Xspace \rightarrow \Yspace$ : \textbf{hard label function};\\ 
% that maps class scores / probabilities to discrete classes.
Typically created by $h(\xv) = \argmax \limits_{k \in \gset} \fkx$ or \\ $h(\xv) = \argmax \limits_{k \in \gset} \pikx$  \\

$\yf$ or $\yfi$: \textbf{margin} for ($i$-th) observation in binary classification\\%\\ classification (with $\Yspace = \{-1, 1\}$). \\

$c \in \R$, s.t. $h(\xv) := [\pix \geq c]$ or $h(\xv) := [\fx \geq c]$: \textbf{threshold} for hard label assignment in binary case (common: $c = 0$ for scoring, $c = 0.5$ for probabilistic classifiers)

% \hrule width7cm
\vspace{1cm}

$\yh$, $\fh$, $\hh$, $\pikxh$, $\pixh$ and $\thetah$ \\
The hat symbol denotes \textbf{learned} functions and parameters.

\end{myblock}

%-------------------------------------------------------------------------------
% Loss and Risk 
%-------------------------------------------------------------------------------
\begin{myblock}{Loss, Risk and ERM}
  $L: \Yspace \times \R^g \to \R^+_0$ : \textbf{loss function}: 
 Quantifies "quality" $\Lxy$ of prediction $\fx$ (or $\Lpixy$ of prediction $\pix$) for true $y$. \\

\textbf{(Theoretical) risk:} $\risk:  \Hspace \to \R $, 
$\riskf = \E_{(\xy \sim \Pxy)}[\Lxy]$
  
\textbf{Empirical risk:} $\riske:  \Hspace \to \R $, 
% summed loss of model over data.
$\riskef = \sumin \Lxyi$,  analogously:
$\riske : \Theta \to \R$; $\risket = \sumin \Lxyit$\\
  
\textbf{Empirical risk minimization (ERM)}:
% -- figuring out which model $\fh$ has the smallest summed loss. \\ 
$\thetavh \in \argmin \limits_{\thetav \in \Theta} \risket$ \\
% $= \argmin \limits_{\thetav \in \Theta} \sumin \Lxyit,$ \\

\textbf{Bayes-optimal model}: $\fbayes = \argmin_{f: \Xspace \rightarrow \R^g}  
\riskf \\

\textbf{Regularized risk}: $\riskr: \Hspace
\to \R, \riskrf = \riskef + \lambda \cdot J(f)$ with regularizer $J(f)$, 
complexity control parameter $\lambda > 0$ (analogous for $\thetav$).

\end{myblock}





% End Second Column					
%%%%%%%%%%%%%%%%%%%%%%%%%%%%%%%%%%%%%%%%%%%%%%%%%%%%%%%%%%%%%%%%%%%%%%%%%%%%%%%%
				}
			\end{minipage}
		\end{beamercolorbox}
	\end{column}
	\begin{column}{.31\textwidth}
		\begin{beamercolorbox}[center]{postercolumn}
			\begin{minipage}{.98\textwidth}
				\parbox[t][\columnheight]{\textwidth}{
%%%%%%%%%%%%%%%%%%%%%%%%%%%%%%%%%%%%%%%%%%%%%%%%%%%%%%%%%%%%%%%%%%%%%%%%%%%%%%%%
% Begin Third Column#


%-------------------------------------------------------------------------------
% Regression Losses 
%------------------------------------------------------------------------------- 
\begin{myblock}{Regression Losses}
  \textbf{L2 loss / squared error:} 
\begin{itemize}    
  \setlength{\itemindent}{+.3in}
  \item $\Lxy = (y-\fx)^2$ or $\Lxy = 0.5 (y-\fx)^2$
  \item Convex and differentiable, non-robust against outliers
  % \item Tries to reduce large residuals (loss scaling quadratically)
  \item Optimal constant model: $\fxh = \meanin \yi =
  \bar{y}$
  \item Optimal model over $\Pxy$ for unrestricted $\Hspace$: $\fxh = \E[y | \xv]$
  % \item $\fxh = \text{mean of } y | \bm{x}$
\end{itemize}

\vspace*{1ex}
%        \includegraphics[width=1\columnwidth]{img/reg_loss.PNG}


  \textbf{L1 loss / absolute error:} 
\begin{itemize}
\setlength{\itemindent}{+.3in}
  \item $\Lxy = |y-\fx|$
  \item Convex and more robust, non-differentiable
  \item Optimal constant model: $\fxh = \text{med}(y^{(1)}, \ldots, y^{(n)})$
  \item Optimal model over $\Pxy$ for unrestricted $\Hspace$: $\fxh = \text{med} [y | \xv]$
  % \item \textcolor{orange}{Optimal model for unrestricted $\Hspace$: $\fxh = \meanin \text{med}(\yi | \xi)$}
  % \item $\fxh = \text{median of } y | \bm{x}$     
\end{itemize}
  %\includegraphics[width=1.03\columnwidth]{img/reg_loss_2.PNG} 
\end{myblock}

%-------------------------------------------------------------------------------
% Classification Losses 
%------------------------------------------------------------------------------- 

\begin{myblock}{Classification Losses}

% \textbf{0-1 loss} \\
% $\Lhxy = [y \neq \hx]$ ~ for $\Yspace = \setzo$ \\
\textbf{0-1-loss (binary case)}\\
$L (y, h(\xv)) = \I(y \neq h(\xv))$\\
$L (y, \fx) = \I( y\fx < 0)$ for $\Yspace = \setmp$ \\ 
Discontinuous, results in NP-hard optimization\\
%Optimal constant model: $h(\xv) \in \argmax \limits_{j \in {0,1}} \sumin \I(\yi = j) $\\

\textbf{Brier score (binary case)} \\
$\Lpixy = (\pix - y)^2$ for $\Yspace = \setzo$ \\
Least-squares on probabilities\\
%Optimal constant model: $\pixh = \bar{y}$\\


\textbf{Log-loss / Bernoulli loss / binomial loss (binary case)}\\
$\Lpixy = -y \log(\pix) - (1-y) \log(1-\pix)$ for $\Yspace = \setzo$ \\
$\Lpixy = \log(1 + (\frac{\pix}{1-\pix})^{-y})$ for $\Yspace = \setmp$ \\
%Optimal constant model: $\pixh = \bar{y}$\\

Assuming a logit-link $\pix = \exp(\fx) / ( 1+\exp(\fx))$:\\
$\Lxy = -y \cdot \fx + \log(1 + \exp(\fx))$ for $\Yspace = \setzo$ \\
$\Lxy = \log(1 + \exp(- y \cdot \fx))$ for $\Yspace = \setmp$ \\
Penalizes confidently-wrong predictions heavily\\

\textbf{Brier score (multi-class case)} \\
$\Lpixy =  \sumkg (\pikx - o_k(y))^2$ \\
%Optimal constant model: $\pixh = \left(\meanin o_1^{(i)}, \meanin o_g^{(i)}\right)$ \\

\textbf{Log-loss (multi-class case)} \\
$ \Lpixy =  - \sumkg o_k(y) \log(\pikx)$ \\  %\\
%Optimal constant model: $\pixh = \left(\meanin o_1^{(i)}, \meanin o_g^{(i)}\right)$ \\

\underline{Optimal constant models}\\[-1ex]
0-1-loss: $h(\xv) \in \argmax \limits_{j \in {0,1}} \sumin \I(\yi = j) $\\
Brier and log-loss (binary): $\pixh = \bar{y}$\\
Brier and log-loss (multiclass): $\pixh = \left(\meanin o_1^{(i)}, \dots, \meanin o_g^{(i)}\right)$ 

%\textcolor{orange}{ADD PROPERTIES OF LOSSES}

\end{myblock}
%-------------------------------------------------------------------------------
% Classification 
%------------------------------------------------------------------------------- 

%\begin{myblock}{Classification}
% 				    We want to assign new observations to known categories according to criteria learned from a training set.  
%             \vspace*{1ex}
%             

%$y \in \Yspace = \gset : $ categorical output variable (label)\\ 

%\textbf{Classification} usually means to construct $g$ \textbf{discriminant functions}:
  
%$f_1(\xv), \ldots, \fgx$, so that we choose our class as \\ $h(\xv) = \argmax_{k \in \gset} \fkx$ \\

%\textbf{Linear Classifier:} functions $\fkx$ can be specified as linear functions\\

% \hspace*{1ex}\textbf{Note: }All linear classifiers can represent non-linear decision boundaries \hspace*{1ex}in our original input space if we include derived features. For example: \hspace*{1ex}higher order interactions, polynomials or other transformations of x in \hspace*{1ex}the model.

%\textbf{Binary classification: }If only 2 classes ($\Yspace = \setzo$ or  $\Yspace = \setmp$) exist, we can use a single discriminant function $\fx = f_{1}(\xv) - f_{2}(\xv)$.  \\


% \textbf{Generative approach }models $\pdfxyk$, usually by making some assumptions about the structure of these distributions and employs the Bayes theorem: 
% $\pikx = \postk \propto \pdfxyk \pik$. \\ %It allows the computation of \hspace*{1ex}$\pikx$. \\
% \textbf{Examples}: Linear discriminant analysis (LDA), Quadratic discriminant analysis (QDA), Naive Bayes\\
% 
% \textbf{Discriminant approach }tries to optimize the discriminant functions directly, usually via empirical risk minimization:\\ 
% $ \fh = \argmin_{f \in \Hspace} \riske(f) = \argmin_{f \in \Hspace} \sumin \Lxyi.$\\
% \textbf{Examples}: Logistic/softmax regression, kNN


%\end{myblock}

%-------------------------------------------------------------------------------
% HRO - Components of Learning 
%-------------------------------------------------------------------------------          
%\begin{myblock}{Components of Learning}

%\textbf{Learning = Hypothesis space + Risk + Optimization} \\
%\phantom{\textbf{Learning}} \textbf{= }$ \Hspace + \risket + \argmin_{\thetav \in \Theta} 
%\risket$

% 
% \textbf{Learning &= Hypothesis space &+ Risk  &+ Optimization} \\
% &= $\Hspace &+ \risket &+ \argmin_{\thetav \in \Theta} \risket$
% 
% \textbf{Hypothesis space: } Defines (and restricts!) what kind of model $f$
% can be learned from the data.
% 
% Examples: linear functions, decision trees
% 
% \vspace*{0.5ex}
% 
% \textbf{Risk: } Quantifies how well a model performs on a given
% data set. This allows us to rank candidate models in order to choose the best one.
% 
% Examples: squared error, negative (log-)likelihood
% 
% \vspace*{0.5ex}
% 
% \textbf{Optimization: } Defines how to search for the best model, i.e., the model with the smallest {risk}, in the hypothesis space.
% 
% Examples: gradient descent, quadratic programming


%\end{myblock}
% End Third Column
%%%%%%%%%%%%%%%%%%%%%%%%%%%%%%%%%%%%%%%%%%%%%%%%%%%%%%%%%%%%%%%%%%%%%%%%%%%%%%%%
			  }
			\end{minipage}
		\end{beamercolorbox}
	\end{column}
\end{columns}

\end{frame}
\end{document}
